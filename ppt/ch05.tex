\section{非线性积分表达式化简}
\begin{frame}
\frametitle{非线性积分表达式化简}
\[
    a=\int{u_x v \dd x},b=\int{u v_x \dd x},c=uv
\]
完全匹配
\[
    a+b=c
\]
非完全匹配
\[
    a+2b=c+b
\]
线性可约
\[
    a^2+2ab+b^2=c^2
\]
线性不可约
\[
    a^2+3ab+b^2=c^2+ab
\]
\end{frame}

\begin{frame}
嵌套积分
\[
    \int\!{\left(u\cdot\int\!{v_xw\dd x}+u\cdot\int\!{vw_x\dd x}\right)\dd x}=\int\!{uvw\dd x}
\]
积分视为导数
\[
    \int\!{\left(u\cdot\int\!{v\dd x}+\int\!{u\dd x}\cdot v\right)\dd x}=\int\!{u\dd x}\cdot\int\!{v\dd x}
\]
\end{frame}

\begin{frame}
基本元素:
\[
\begin{aligned}
    f&=\underbrace{\int\!\int\!\cdots\!\int}_m{ \frac{\partial^n}{\partial v_1 \partial v_2 \cdots \partial v_n} (g_1 g_2 \cdots g_l)\dd u_1 \dd u_2 \cdots \dd u_m} \\ 
    &=\partial^U_V(G)
\end{aligned}
\]
化简对象:
\[
    I = \sum_{k=1}^N{c_k I_k}
\]
\end{frame}

\begin{frame}
导数规则:
\[
    u_xvw+uv_xw+uvw_x=(uv)_xw+uvw_x=(uvw)_x
\]
乘法规则:
\[
    ab+ac+ad=a(b+c)+ad=a(b+c+d)
\]
都基于二项合并规则
\[
    I_1+I_2=I_3
\]
\end{frame}

\begin{frame}
化简过程:
\[
\begin{aligned}
I &= \sum_{k=1}^N{c_k I_k}-\sum_{l=1}^L{x_l ({J_1}_l+{J_2}_l-{J_3}_l)} \\ 
& = \bm{B}(\bm{b}-\bm{A}\bm{x})
\end{aligned}
\]
优化问题:
\[
    \underset{\bm x}\min\VecNorm{\bm{b}-\bm{A}\bm{x}}_1
\]
线性规划模型:
\[
\begin{aligned}
    &\underset{\bm u,\bm x}\min \sum_{k=1}^L{u_k},\\
    &s.t. \left\{
    \begin{matrix}
    \bm{u}\ge \bm{b}-\bm{A}\bm{x},\\ 
    \bm{u}\ge \bm{A}\bm{x}-\bm{b},
    \end{matrix}
    \right.
\end{aligned}
\]
\end{frame}

\begin{frame}{一个典型的例子}
\[
\renewcommand{\arraystretch}{1.0}
\begin{array}{l}
-20u_xw_{xx}v
+c\ii{u_{xxxxx}\ii{v}}
+c\ii{u_{xxxxx}v}
+c\ii{v_{xxxxx}u}\\% 1
+20u_x^2v_xa
+20u_x\ii{wv_{xxx}}
+20w_x\ii{v_xu_{xx}}
+20w_x\ii{u_xv_{xx}}\\% 2
+10c\ii{v_{xxx}u^2}
+30c\ii{uu_xv_{xx}}
-20u_xwv_{xx}
-60cuu_x\ii{uv}\\% 3
+10c\ii{u_x\ii{u_xv_{xx}}}
+10c\ii{u_x\ii{v_xu_{xx}}}
-30vu^3c
-40u_xw_xv_x\\% 4
+30c\ii{u^2u_{xx}\ii{v}}
+20c\ii{v_xu_{xx}u}
+30c\ii{u_{xxx}u_x\ii{v}}\\% 5
+30c\ii{u_{xx}vu_x}
+40u_x\ii{w_xv_{xx}}
+40u_x\ii{v_xw_{xx}}
-120uu_xwv\\% 6
-4u_x^2v_xb
+20u_x\ii{vw_{xxx}}
+120u_x\ii{vuw_x}
+120u_x\ii{uvw_x}\\% 7
+60c\ii{uu_x^2\ii{v}}
+120c\ii{u^2u_xv}
-cuv_{xxxx}
-10u_{xxx}c\ii{vu}\\% 8
-10u_{xxx}c\ii{u_x\ii{v}}
-20u_xa\ii{u_xv_{xx}}
-20u_xa\ii{v_xu_{xx}}\\% 9
+4u_xb\ii{u_xv_{xx}}
-10cu\ii{u_xv_{xx}}
-10c v u u_{xx}
-10cu\ii{v_xu_{xx}}\\% 10
+4u_xb\ii{v_xu_xx}
+20c\ii{u_{xx}^2\ii{v}}
-cu_{xxxxx}\ii{v}\\% 11
+10c\ii{u_{xxxx}u\ii{v}}
+30u^2u_xc\ii{v}
-60uu_xc\ii{u_x\ii{v}}\\% 12
-20cu_xu_{xx}\ii{v}
+120u_x\ii{vwu_x}
+c\ii{u_xv_{xxxx}}
-10v_{xx}u^2c\\% 13
+30c\ii{u^3v_x}
+20c\ii{vu_{xxx}u}=0\\% 14
\end{array}
\]
\end{frame}

\begin{frame}{找到所有合并规则再化简的理由}
\[
\begin{array}{l}
0=\red{u_x\int\!{v_{xx}w_x\dd x}}+\red{u_x\int\!{v_xw_{xx}\dd x}}-\red{u_xv_xw_x}\\
~~+w_x\int\!{u_xv_{xx}\dd x}+w_x\int\!{u_{xx}v_x\dd x}-\red{u_xv_xw_x}\\
~~+u_x\int\!{vw_{xxx}\dd x}+\red{u_x\int\!{v_xw_{xx}\dd x}}-u_xvw_{xx}\\
~~+u_x\int\!{v_{xxx}w\dd x}+\red{u_x\int\!{v_{xx}w_x\dd x}}-u_xv_{xx}w ,
\end{array}
\]
如果事先合并了
\[
    u_x\int\!{v_xw_{xx}\dd x}+u_x\int\!{v_{xx}w_x\dd x}=u_xv_xw_x
\]
则无法继续合并剩余部分
\end{frame}

\begin{frame}{嵌套积分的多样性}
内部因式分解
\[
    c\int\!{\int\!{v\dd x}\,uu_{xxxx}\dd x}+c\int\!{\int\!{v\dd x}\,u_xu_{xxx}\dd x}=c\int\!{\int\!{v\dd x}\,(uu_{xxx})_x\dd x}
\]
积分视为导数
\[
    c\int\!{\int\!{v\dd x}\,(uu_{xxx})_x\dd x}+c\int\!{vuu_{xxx}\dd x}=cuu_{xxx}\int\!{v\dd x}
\]
\end{frame}



\begin{frame}{化简能力对比}
\begin{enumerate}
\item 积分项内部的乘法项不少于2个.
\item 表达式含有不完全匹配的积分项.
\item 需要将一些嵌套积分视为导数. 
\item 嵌套积分能够被递归化简.
\item 积分多项式是线性不可约的.
\item 抽象函数在非线性函数内部, 如$u^{-1}$和$\sin(u)$等.  
\item 表达式含有具体函数, 如$x$和$\sin(x)$等. 
\end{enumerate}
\end{frame}

\begin{frame}
\begin{table}
\centering
\begin{tabular}{ccl}
\hline
积分类型 & 能否化简 & \multicolumn{1}{c}{表达式} \\
\hline
0000000 & 111 & $I_1=\int\!{(uv)_x\dd x}$\\ 
1000000 & 111 & $I_2=\int\!{(u^2v)_x\dd x}$\\ 
0100000 & 001 & $I_3=I_1+\int\!{u_xv\dd x}$\\ 
0010000 & 001 & $I_4=\int\!{(\int\!{u\dd x}\cdot \int\!{v\dd x})_x\dd x}$\\
0001000 & 001 & $I_5=\int\!{u\cdot \int\!{(vw)_x\dd x}\dd x}$\\
0000100 & 001 & $I_6=I_1^2+\int\!{(vw)_x\dd x}$\\
0000010 & 100 & $I_7=\int\!{(\sin(u)\sqrt{v})_x\dd x}$\\
0000001 & 110 & $I_8=\int\!{(\sin(x)u)_x\dd x}$\\
1111100 & 001 & $I_9=\int\!{(\int\!{u\dd x}\int\!{(\int\!{uv\dd x}\;w)_x\dd x\dd x})_x\dd x}+I_3^2$\\
1111111 & 000 & $I_{10}=I_9+\int\!{(\sin(x)/u)_x\dd x}$\\
\hline
\end{tabular}
\end{table}
\end{frame}