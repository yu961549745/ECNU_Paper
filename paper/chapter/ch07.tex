\chapter{总结与展望}\label{ch07}
\section{总结}
非线性系统是描述自然现象及其内在规律的主要数学模型, 其研究在众多领域内都发挥着重要的作用. 本文基于符号计算平台 Maple, 针对非线性系统的精确解和相关基础算法开展了以下工作.

(一) 非线性微分方程精确求解的机械化算法

在非线性微分方程求解的部分, 我们首先整合了\Painleve{}展开法\D 简单 Hirota 方法\D 共轭参数法和长极限方法, 研发了自动推导非线性演化方程的孤子解\D 呼吸子解和 lump 解的软件包 TwSolver. 在研究过程中, 通过引入参数下标集合将$n$孤子解公式推广到不可积方程, 由此可获得不可积方程的真解. 同时, 在编程实现中, 我们对相关方法的细节进行了一些优化. 例如:
\begin{compactenum}[(1)]
\item 在\Painleve{}展开法中, 我们提出了一个递归求解算法来优化待定函数的求解过程.
\item 在 Hirota 方法中, 优化了多孤子解的生成公式.
\item 在长极限方法中, 推导了关键参数的计算方法, 还优化了 lump 解的生成公式.
\item 最终实现的软件包对解的推导\D 验证和作图都提供了方便易用的接口. 
\end{compactenum}

此外, 针对直接代数方法中大规模非线性代数方程组求解困难的问题, 我们提出了一个分组并行的求解算法, 并将其实现为 PGSolve 软件包. 我们以求$n$孤子-1 lump 相互作用解的直接代数方法为例, 基于 PGSolve 开发了 NS1L 软件包. 在 NS1L 软件包中, 我们优化了解的生成公式, 推导出了解的筛选条件. 实验表明, 采用继承求解的方式, PGSolve 能够在几分钟内求解 NS1L 产生的含300个左右方程的方程组, 其效率是普通求解函数的 100 倍以上.  

(二) $n$阶展开方法及其应用研究

\red{

本文将齐次平衡原则推广为$n$阶展开方法, 并将其应用于微分方程和差分方程的若干求解方法中. 本文通过同时平衡方程中最高$n$项的阶数和系数, 解决了齐次平衡原则在某些情况下无法确定解的阶数的上界的问题, 提出了$n$阶展开方法, 并研发了相应的软件包 NEM. 基于$n$阶展开方法, 我们提出了一个求解非线性差分方程多项式解的新算法, 并研发了相应的软件 NLREPS. 同时, 本文还将$n$阶展开方法应用于双曲正切方法和\Painleve{}展开法, 开发了相应的软件包 NCTM 和 PExpand. 实验和例子表明, $n$阶展开方法从理论上对齐次平衡原则进行了完善, 能更加全面地分析平衡的情况, 在求解时获得更多的解. 此外, $n$阶展开方法对\Painleve{}展开法的完善也使得本文开发的 TwSolver 和 NS1L 能够求解以往不能求解的方程. 

}

(三) 非线性积分表达式化简

因为在非线性微分方程求解的过程中往往需要进行非线性积分表达式的化简, 本文将非线性积分化简作为一个具有挑战性的问题进行研究. 本文基于导数的乘法规则提出了一个递归算法来寻找所有的二项合并规则, 基于这些规则, 将积分表达式化简的问题转化为一个精确线性规划问题进行求解. 最终, 本文实现的 IntSimplify 软件包能够化简线性不可约且含冗余项和嵌套积分的积分多项式, 解决了许多现有算法不能解决的问题. 实验表明, 我们的算法在实际应用中往往是高效的. 

\section{展望}
基于本文的工作, 还有以下几方面的内容值得深入研究.
\begin{compactenum}[(1)]
\item 本文的 TwSolver 以\Painleve{}展开法为基础确定方程的变换, 事实上这是对数变换的推广. 有的方程并不存在此类变换, 还有有理变换\D 复合函数变换等更多形式的变换, 有待进一步考虑.
\item 不管是在 TwSolver 中还是在 NS1L 中, 如果能够将变换后的方程转化为双线性形式, 就能极大地简化问题\D 提高求解效率. 虽然已经有一些关于双线性的机械化工作, 但它们的适用范围都不是很广. 因此, 开发一个更加通用的双线性变换的软件包也是非常有意义的研究工作.
\item 本文开发的 NLREPS 软件包能够实现非线性差分方程多项式解的求解. 如果能够参照线性多项式系数差分方程的发展路线, 继续求解有理函数解\D 超几何函数解等一系列更加一般的解, 也是一个非常有意义的工作. 
\item 本文提出的$n$阶展开方法很好地完善了齐次平衡原则, 但是目前只适用于单个方程的求解. 对于含多个未知函数的方程组, 第二类和第三类平衡点的确定就变得十分困难. 因此, 将$n$阶展开方法推广到方程组也是值得研究的工作.
\item 最后, 本文所实现的 IntSimplify 软件包只考虑了导数的乘法规则, 没有考虑导数的除法规则和复合求导的规则. 将剩下的两个规则纳入考虑范围之后, 再将问题转化为精确线性规划进行求解, 就能化简更加一般的积分表达式, 这一部分工作我们将后续进一步推广研究. 
\end{compactenum}
