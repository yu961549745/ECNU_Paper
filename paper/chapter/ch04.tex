\chapter{非线性积分表达式化简}
\section{Introduction}\label{Introduction-03}

Symbolic integration is one of the most important components in a computer algebra system. For the symbolic integration of elementary functions, the research had been started in the early stage of symbolic computation. The corresponding integral softwares SAINT \spell{(Symbolic Automated INTegrator)} and SIN \spell{(Symbolic INtegrator)} were published in 1963 \citep{slagle1963} and 1967 \citep{moses1967}, respectively. The most important algorithm, which laid the foundation for the development of symbolic integration in recent decades, is proposed by Risch \citep{risch1969,risch1970}. Moses firstly implemented the case of purely transcendental functions in the improved SIN soon after Risch Algorithm had been published \citep{moses1971}. The general case of elementary functions in Risch Algorithm was solved and implemented by Bronstein in 1990 \citep{bronstein1990}. Almost all computer algebra systems, such as Maple, Mathematica, Axiom, Maxima and Reduce, are using appropriate extended Risch Algorithm to perform symbolic integration. After the integration of elementary functions was almost solved, researchers mainly focused on the integration of special functions \citep{cherry1985,cherry1986,bertrand1994,jeffrey1997}.

In recent years, researchers mainly focus on special types of integrals, such as relativistic coulomb integrals \citep{paule2012,paule2013}, Feynman integrals \citep{blumlein2012,smirnov2015} and parameter integrals \citep{raab2016}.


However, there are few studies on the integral of abstract functions. Only \cite{deconinck2009} and \cite{poole2010} proposed an algorithm of finding integral of linear and integrable expressions of abstract functions. Their algorithm, which can solve the equation $D_x f=g$, is mainly designed for finding the conservation laws of nonlinear partial differential equations \citep{poole2011}, not for the general integral of abstract functions. So this paper aims to solve the integral of abstract functions in general cases, and compare our algorithm with those designed for general integrals in the computer algebra systems such as Maple and Mathematica.

In Maple and Mathematica, integrals will be simplified only when the inside of an integral is the differential of another expression. For example, 
\begin{equation}
\int\!{u_xv+uv_x \dd x}=\int\!{(uv)_x\dd x}=uv.
\label{complete_matched}
\end{equation}
We say such integrals are \emph{complete matched}, which is also called \emph{integrable} in some other articles. In addition, Maple can only handle single variable functions while Mathematica can handle multi-variable functions. 

Otherwise, if there are surplus items inside an integral, we say it is \emph{incomplete matched}. For example,
\begin{equation}
\int\!{u_xv+2uv_x \dd x}=uv+\int\!{uv_x\dd x}.
\label{incomplete_matched}
\end{equation}
Neither Maple nor Mathematica can do such simplifications. 

By using the combination functions, Maple and Mathematica can simplify linear combination of integrals that are \emph{complete matched}. But for polynomials of integrals, problem would be much more difficult. 

If a polynomial can be factorized into a multiplication of linear expressions, we say the polynomial is \emph{linear reducible}. For these polynomials, we can use factorization algorithm to reduce the problem into linear cases that can be simplified by Maple or Mathematica. For example, let $a=\int{u_x v \dd x},b=\int{u v_x \dd x},c=uv$, we can get a \emph{linear reducible} expression
\begin{equation}
a^2+2ab+b^2=(a+b)^2=c^2.
\label{liner_reducible}
\end{equation}

But in general cases, polynomials are not always \emph{linear reducible}. For example, under the same assumption as  \refeqn{liner_reducible}, we have
\begin{equation}
a^2+3ab+b^2=(a+b)^2+ab=c^2+ab.
\label{non_linear_reducible}
\end{equation} 
We say it is \emph{linear unreducible}. There is no algorithm can do such combinations. It is more difficult than factorization, because the result is not unique. In other words, we cannot simplify such expressions by the built-in algorithms in Maple or in  Mathematica. 

Problems would be more complicated for the expressions that contain nested integrals, such as 
\begin{equation}
\int\!{\left(u\cdot\int\!{v_xw\dd x}+u\cdot\int\!{vw_x\dd x}\right)\dd x}=\int\!{uvw\dd x}.
\label{nested_integral}
\end{equation}
Maple and Mathematica cannot do such simplifications because they cannot factorize the inside of an integral automatically. Sometimes, a nested integral should be regarded as  a differential. For example,
\begin{equation}
\int\!{\left(u\cdot\int\!{v\dd x}+\int\!{u\dd x}\cdot v\right)\dd x}=\int\!{u\dd x}\cdot\int\!{v\dd x}.
\label{integral_as_differential}
\end{equation}
Maple and Mathematica cannot process such expressions either. Furthermore, \emph{incomplete matched} and \emph{linear unreducible} expressions might occur recursively inside an integral, which would make the simplification getting harder and harder.

In summary, based on the built-in algorithms in Maple and Mathematica, we can only simplify \emph{linear reducible} polynomials of simple integrals that are \emph{complete matched}. For more general cases with arbitrary polynomials of complex integrals that might be \emph{incomplete matched}, we propose a new algorithm to solve them. The algorithm is packed in a Maple package named \verb|IntSimplify|, which has two exported procedures \verb|IntExpand|  and \texttt{IntCombine}. The procedure \verb|IntExpand| aims to expand the input expression to the standard form that accepted by our simplification procedure \texttt{IntCombine}. 

The paper is organized as follows. In \refsec{Definitions-03} we introduce some definitions to better describe the integral simplification problem. In \refsec{Simplify-03} we introduce the framework of our algorithm. Then, key algorithms would be described in \refsec{Details-03}. We will analyze the capability and time complexity in \refsec{Results-03}. Finally, conclusions are given in \refsec{Conclusion-03}.

\section{Definitions}\label{Definitions-03}
An abstract integral expression (AIE) is generated by applying the following operations on a set of abstract functions (such as $u(x,y)$):
\begin{compactenum}[(1) ]
\item A derivative w.r.t. an independent variable;
\item An integral of the above;
\item A product of the above;
\item A linear combination of the above;
\item Any iteration of the above operations.
\end{compactenum}
Thus making this an algebra.

A standard integral expression (SIE) is an expression that expanded from an AIE that has no addition inside it. The formal definition can be represented as follows. Let $\mathcal V$ represent a set of variables, and $\mathcal F$ represent a set of continuous abstract functions that defined on $\mathcal V$ or its subsets. Especially, we regard an abstract constant as an abstract function with an empty variable set, which is also contained in the set $\mathcal F$. Applying  multiplication, integral and differential on elements of $\mathcal F$, generates a \emph{closed set} of $\mathcal F$ denoted as $\mathbb F$. A SIE is
\begin{equation}
I = \sum_{k=1}^N{c_k I_k},
\label{std_form}
\end{equation} 
where $c_k \in K$ and $I_k \in \mathbb F$. It means that $I_k$ has the same form as 
\begin{equation} 
f =\underbrace{\int\!\int\!\cdots\!\int}_m{ \frac{\partial^n}{\partial v_1 \partial v_2 \cdots \partial v_n} (g_1 g_2 \cdots g_l)\dd u_1 \dd u_2 \cdots \dd u_m},
\label{int_form}
\end{equation} 
where $u_1,\cdots,u_m,v_1,\cdots,v_n \in \mathcal V$. Especially, $g_1,\cdots,g_l \in \mathcal F$ or they have the same form as $f$. We say $f$ is a standard integral item (SII). It is defined by  \refeqn{int_form} recursively.

The definition of SIE not only covers ordinary integrals and differentials, but also covers multiplication of them, even covers nested integrals. Due to the linearity of integral and differential, an AIE can be transformed into a SIE by expansion. We implement the procedure \verb|IntExpand| to do it. 

Since the continuity assumption of abstract functions, integrals and differentials can be switched in any order. Similarly, the multiplication of functions is also independent of order. In order to give a more concise representation of SII as well as show the disorder of these operations, we rewrite \refeqn{int_form} as
\begin{equation} 
f=\partial^U_V(G). \label{st_form}
\end{equation} 
Here,  $U=[u_1,\cdots,u_m], V=[v_1,\cdots,v_n] \text{~and~} G=[g_1,\cdots,g_l]$ are unordered lists of integral variables, differential variables and integrands, respectively. This form has several shorthands in this paper, when $U,V$ are empty or only contain one element, such as 
\begin{equation}
    \partial_x=\partial^\varnothing_{[x]} \text{,~} \partial^x=\partial_\varnothing^{[x]} \text{~and~} \partial(G)=\partial^\varnothing_\varnothing(G).
\end{equation}
The last one is used to distinguish between integral item $\partial(G)$ and unordered list $G$. 

In the representation of  \refeqn{st_form}, an integral item is clearly divided into three components. In order to get these components from an integral item $f$, we define the following operations:
\begin{equation}
IV(f)=U,DV(f)=V,FC(f)=G.
\end{equation}
At the same time, if $f$ contains only one abstract function we define $TP(f)=simple$ , otherwise $TP(f)=complex$. The \emph{common variable set} of an integral is also important to our problem. We define 
\begin{equation}    
FV(f)=\left\{
\begin{array}{cl}
\bigcap\limits_{g\in FC(f)}{FV(g)}, &\text{if }TP(f)=complex;\\ 
\text{variable set of } f,          &\text{if }TP(f)=simple.
\end{array}
\right.
\end{equation}  

In order to perform set operations on these unordered lists, we regard them as multisets. We can implement operations such as multiplication, integral and differential based on the operations on multisets, i.e., 
\begin{equation}
\begin{split}
\partial(G_1)\cdot\partial(G_2)&=\partial(G_1+G_2),\\
\partial^{U'}_{V'}\partial^{U}_{V}(G)&=\partial^{U+U'}_{V+V'}(G).
\end{split}
\label{ops}
\end{equation}
Here, $+$ means the set union operation. Applying these operations recursively can compress the structure of SIIs. Thus, $TP(f)=complex$ implies that $|FC(f)|\ge 2$. 

Based on the definitions above, we can represent a polynomial of integrals as a linear combination of SIIs. The multiplication, integral and differential operations on SIIs can be implemented by the operations of multisets as shown in  \refeqn{ops}. 

\section{The framework of our algorithm} \label{Simplify-03}
The goal of our algorithm is to rewrite a given AIE and minimize its complexity. A set of equivalences that called combination rules is founded in \refsec{all_rules-03}. The objective is to apply them in such a way as to minimize the size of the result. The actual optimization is cast as an $\ell_0$-norm optimization. Since this is NP-hard, instead the common surrogate $\ell_1$-norm optimization is performed in \refsec{optimization-03}.


\subsection{Finding all combination rules}\label{all_rules-03}

In general, combination rules have so many possible forms, such as $u_x v + u v_x = (uv)_x$, $u_x v w+u v_x w + u v w_x = (uvw)_x$ and so on. It is impossible to enumerate all of them. But consider it reversely, the differential rules of multiplications with more than two items can be inferred by the rules of two items. For example, $u_xvw+uv_xw+uvw_x=(uv)_xw+uvw_x=(uvw)_x$.  

For the simplest combination rule $u_x v + u v_x = (uv)_x$, we can do multiplication, integral and  differential on it to construct new rules. Ignore the constant coefficients of rules, we would get the rules in the form of $I_1+I_2=I_3$, where $I_1,I_2,I_3\in \mathbb F$. As a result, we can obtain all possible combination rules in the form of $I_1+I_2=I_3$ by using the differential rule of multiplication with two items.

Let $\mathcal I =\{I_1,\cdots,I_N\}$ to be the set of SIIs in \refeqn{std_form}. Let set $\mathbb I$ satisfy $\mathcal I \subset \mathbb I$, we say set $\mathbb I$ is the \emph{generated set} of $\mathcal I$, if $I_1,I_2\in \mathbb I$ and $I_1+I_2=I_3$, then $I_3\in \mathbb I$. Assume that $\mathbb I=\{B_1,\cdots,B_M\}$, let all possible combination rules from the set $\mathbb I$ to constitute a set $\mathcal R=\{{J_1}_l+{J_2}_l={J_3}_l|l=1..L\}$, where ${J_1}_l,{J_2}_l,{J_3}_l \in \mathbb I$.

% 换成纯文字描述节省篇幅
The algorithm of establishing $\mathcal R$ can be described as Algorithm \ref{FindAllRules}. Here, $FindRulesIn(I)$ means find rules in the set $\{(I_1,I_2)|I_1,I_2,\in I\}$, and $FindRulesBetween(I,J)$ means find rules in the set $\{(I_1,I_2)|I_1\in I, I_2\in J\}$.

\begin{algorithm}
\caption{Finding all combination rules.}
\label{FindAllRules}
\KwIn{A SIE $e$.}
\KwOut{Set of all combination rules.}
\Fn{$FindAllRules(e)$}{
    $I\gets$ set of SIIs in $e$; $R\gets FindRulesIn(I)$\;
    \While{$\bf{true}$}{
        $J\gets$ new SIIs in $R$\;
        \lIf{$J=\emptyset$}{$\bf{break}$}
        $R\gets R \cup FindRulesIn(I) \cup FindRulesBetween(I,J)$\;
    }
    \Return{$R$}\;
}
\end{algorithm}

The major complexity of our algorithm is contributed by the Algorithm \ref{FindAllRules}. Its complexity is directly dependent on the number of elements in the generated set $\mathbb I$ denoted as $N$. Then, the complexity of our algorithm is $\mathcal O(N^2)$. The value of $N$ cannot be obtained from the input expression directly, but it is easy to be estimated at the worst cases. Consider 
\begin{equation}
I_n=\int\!{(f_1\cdot f_2\cdots f_n)_x \dd x},
\label{worst_case}
\end{equation}
where $f_1,f_2,\cdots,f_n$ are distinct functions. It has $n$ integral items after expansion, and these items make up the set $\mathcal I$. Each nonempty subset of $\mathcal I$ can be combined as a distinct integral item of the set $\mathbb I$, which means $N=2^n-1$. So, the complexity of our algorithm is $\mathcal O(4^n)$. Therefore, we can define the order of an input expression as the maximum number of distinct functions multiplied in an integral item. The complexity grows exponentially about the order of input expression.

\subsection{The corresponding optimization problem}\label{optimization-03}

Once the set $\mathcal R$ of all combination rules is established, simplifying an integral expression can be represented as an identity transformation 
\begin{equation}
I=\sum_{k=1}^N{c_k I_k}-\sum_{l=1}^L{x_l ({J_1}_l+{J_2}_l-{J_3}_l)},
\label{normal_simplify}
\end{equation}
where the rules ${J_1}_l+{J_2}_l={J_3}_l(l=1..L)$ belong to $\mathcal R$. Then, the simplification problem can be transformed into a problem of finding the best values of the undetermined coefficients $x_l(l=1..L)$. 

In general, simplification aims to minimize the number of remaining items or to simplify the coefficients of results expressions. These two measures can be better represented when transforming  \refeqn{normal_simplify} into a matrix form.

Assume that 
\begin{equation}
\begin{split}
\sum_{k=1}^N{c_k I_k} &= \sum_{k=1}^M{b_k B_k},\\
\sum_{l=1}^L{x_l ({J_1}_l+{J_2}_l-{J_3}_l)} &= \sum_{j=1}^L{x_j \sum_{i=1}^M{a_{ij} B_i}}.
\end{split}
\end{equation} 
Let $\bm B=(B_1,\cdots,B_M),\bm b=(b_1,\cdots,b_M)\TT,\bm x=(x_1,\cdots,x_L)\TT,\bm A=(a_{ij})_{M \times L}$, \refeqn{normal_simplify} is equals to 
\begin{equation}
I=\bm{B}\bm{b}-\bm{B}\bm{A}\bm{x}=\bm{B}(\bm{b}-\bm{A}\bm{x}).
\end{equation}

To minimize the number of remaining items, we can do $\ell_0$-optimization as 
\begin{equation}
    \underset{\bm x}\min~\VecNorm{\bm{b}-\bm{A}\bm{x}}_0,
\end{equation}
where $\VecNorm{\bm x}_0$ is the $\ell_0$-norm. Since this is NP-hard, instead the common surrogate $\ell_1$-norm optimization is performed, i.e.,
\begin{equation}
\underset{\bm x}\min\VecNorm{\bm{b}-\bm{A}\bm{x}}_1.
\label{LAE}
\end{equation}
In this way, we can not only minimize the number of remaining items approximately, but also simplify the coefficients. 

Fortunately, $\ell_1$-optimization can be transformed into a linear programming problem 
\citep[pp. 195--196]{L1_regression}, i.e.,
\begin{equation}
\begin{split}
&\underset{\bm u,\bm x}\min \sum_{k=1}^L{u_k},\\
&s.t. \left\{
\begin{matrix}
\bm{u}\ge \bm{b}-\bm{A}\bm{x},\\ 
\bm{u}\ge \bm{A}\bm{x}-\bm{b},
\end{matrix}
\right.
\end{split}
\label{LP}
\end{equation}
where $\bm u=(u_1,\cdots,u_L)\TT$. These constraints have the effect of forcing $\bm u = |\bm{b}-\bm{A}\bm{x}|$ upon being minimized, so the objective function is equivalent to the original objective function. Since this problem does not contain the absolute value operator, it is in a format that can be solved by any linear programming algorithm.

Finally, we regard the simplification problem as a linear programming problem. 


\section{Key Algorithms} \label{Details-03}
In this section, the details of finding combination rules would be expounded in \refsec{Combine-03}, and the details of optimization methods would be described in \refsec{optMethods-03}.

\subsection{Algorithm of finding combination rules} \label{Combine-03}

All possible combination rules can be found based on the algorithm of finding combination rules between two different SIIs. The details of this algorithm will be described in this section. 

For the simplest combination rule $u_x v + u v_x = (uv)_x$, we can do multiplication, integral and  differential on it to construct new rules. Hence, for any combinable pair, the combination rule can be reduced to the simplest form that contains only two abstract functions. 

Assume that $I_1,I_2$ are SIIs, consider the combination 
\begin{equation}
\begin{split}
I_1+I_2 &= \partial^U_V(j_1) + \partial^U_V(j_2) \\
        &= \partial^U_V( j_1+j_2 )\\
        &= \partial^U_V( f\cdot(j_3+j_4) )\\ 
        &= \partial^U_V( f\cdot r ) = I_3 .
\end{split}
\label{combine_form}
\end{equation} 

The second equality holds when $I_1,I_2$ have the same integral and differential variables, i.e., $IV(I_1)=IV(I_2),DV(I_1)=DV(I_2)$. The common part between $j_1$ and $j_2$ is extracted and denoted as $f$ in the third equality, and the rest parts are denoted as $j_3,j_4$, respectively.

If the third equality holds outside an integral, it represents the situations like 
\begin{equation}
f\cdot\int\!{u_x v\dd x}+f\cdot\int\!{u v_x \dd x} = f\cdot(uv).
\end{equation} 
Since $f$ is arbitrary, when taking $f=1$, it represents the combination of linear cases; when taking $f$ as a multiplication of integrals, it represents the combination of polynomials. It means that our algorithm is capable of processing polynomial of integrals. 

If the third equality holds inside an integral, it represents the situations like 
\begin{equation}
\int\!{f \left(\int\!{u_x v\dd x}\right)\dd y}+\int\!{f \left(\int\!{u v_x \dd x}\right) \dd y} = \int\!{fuv\dd y}.
\end{equation}
It means that our algorithm also works well for nested integrals. 

The last combination in \refeqn{combine_form} that $j_3+j_4=r$ can be considered by two cases: \emph{root case} and \emph{recursive case}. 

The \emph{root case} means that $j_3+j_4$ can be combined by the most basic differential rule with two items, i.e., 
\begin{equation}
\begin{array}{rl}
& j_3+j_4=u \cdot v+\partial^x(u)\cdot \partial_x(v) = \partial^x(u)\cdot v, \\
\text{or}& j_3+j_4=u \cdot v+\partial_x(u)\cdot \partial^x(v) = \partial^x(v)\cdot u.
\end{array}
\label{root_form}
\end{equation}

The \emph{recursive case} means that $j_3+j_4$ can be combined  finally by using  \refeqn{combine_form} recursively. It implies that $j_3,j_4$ are integrals, not the multiplication of integrals.

For these two cases, it is required that the inner part of integrals is finally multiplied by two parts. So it requires $TP(I_1)=TP(I_2)=complex$.  

Based on the above analysis, we introduce the Algorithm \ref{FindRuleForPair}. For the input pair $I_1,I_2$, we check the common conditions of \emph{root case} and \emph{recursive case}. Then, we fetch $J_1,J_2$ as the inner parts of $I_1,I_2$, respectively. After that, we use set operations to compute the common part of $J_1$ and $J_2$, and denote the different parts of $J_1,J_2$ as $J_3,J_4$, respectively. Finally, we try to combine $J_3,J_4$ directly. If it fails and $|J_3|=|J_4|=1$, we would try to combine them recursively.  

\begin{algorithm}
\caption{Finding combination rules for two integral items.}
\label{FindRuleForPair}
\KwIn{IntFunc object $I_1,I_2$, external integral variables  $S$.}
\KwOut{Return $I_3$ if $I_1 + I_2 = I_3$, or $FAIL$ if $I_1,I_2$ cannot be combined.}
\Fn{$FindRuleForPair(I_1,I_2,S)$}{
    \tcp{Check the common condition.}
    \lIf{\bf{not} $( TP(I_1)=TP(I_2)=complex$  \bf{and} $IV(I_1)=IV(I_2)$ \bf{and} $DV(I_1)=DV(I_2) )$}{
        \Return{$FAIL$}
    }
    \tcp{Initialization}
    $U \gets IV(I_1), V\gets DV(I_1),S\gets U \cup S $\;
    $J_1\gets FC(I_1), J_2\gets FC(I_2) $ \;
    $F\gets J_1 \cap J_2, J_3\gets J_1-F, J_4\gets J_2-F$\;
    \tcp{Ensure $|J_3|\le |J_4|$}
    \lIf{$|J_3|>|J_4|$}{
        $J_3,J_4\gets J_4,J_3$
    }
    \tcp{Try combine directly.}
    \uIf{$\min(|J_3|,|J_4|)=1$}{
        $r,F\gets FindRuleForVars1(\partial(J_3),\partial(J_4),F,S)$\;
    }\uElseIf{$\min(|J_3|,|J_4|)=2$}{
        $r\gets FindRuleForVars2(\partial(J_3),\partial(J_4),S)$\;
    }\lElse{
        $r\gets FAIL$
    }
    \tcp{Try combine recursively.}
    \If{$r=FAIL$ \bf{and} $|J_3|=|J_4|=1$}{
        $r\gets FindRuleForPair(\partial(J_3),\partial(J_4),S) $\;
    }
    \tcp{Build rule.}
    \lIf{$r\neq FAIL$}{
        \Return{$\partial^U_V(\partial(F)\cdot r)$}
    }
    \Return{$FAIL$}\;
}
\end{algorithm}

\begin{algorithm}
\caption{Finding combination rules for normal root cases.}
\label{FindRuleForVars2}
\Fn{$FindRuleForVars2(I_1,I_2,S)$}{
    $[u,v]\gets FC(I_1)$\tcp*[l]{Extract contents of multiset.}
    \For{$x\in S\cap FV(I_1) \cap FV(I_2)$}{
        \lIf{$\partial_x(u) \cdot \partial^x(v)=I_2$}{
            \Return{$\partial_x(u \cdot \partial^x(v) )$}
        }
        \lElseIf{$\partial^x(u) \cdot \partial_x(v)=I_2$}{
            \Return{$\partial_x(\partial^x(u) \cdot v )$}
        }
    }
    \Return{$FAIL$}\;
}
\end{algorithm}

\begin{algorithm}
\caption{Finding combination rules for special root cases.}
\label{FindRuleForVars1}
\Fn{$FindRuleForVars1(u,t,F,S)$}{
    \lIf{$TP(u)\neq complex$}{
        \Return{$FAIL,F$}
    }
    $G \gets FC(u) \cap F$\;
    \For{$v \in G$}{
        $I_1\gets u \cdot v,I_2 \gets t\cdot v$\;
        \For{$x \in S\cap FV(I_1) \cap FV(I_2)$}{
            \uIf{$\partial_x(u) \cdot \partial^x(v)=I_2$}{
                $F\gets F-[v]$\;
                \Return{$\partial_x(u \cdot \partial^x(v) ),F$}\;
            }
            \ElseIf{$\partial^x(u) \cdot \partial_x(v)=I_2$}{
                $F\gets F-[v]$\;
                \Return{$\partial_x(\partial^x(u) \cdot v ),F$}\;
            }
        }
    }
    \Return{$FAIL,F$}\;
}
\end{algorithm}

There are three major differences between the above mathematical description and the algorithmic description in the pseudo-code. 

Firstly, we use multisets instead of SIIs to perform calculations. The symbols of multisets are capitals of integral items. 

Secondly, the \emph{root case} is divided into two sub cases due to the flexibility of collection. 

For the normal case $|J_3|=|J_4|=2$, we can test the two possibilities directly as  \refeqn{root_form}. Sometimes, 
\begin{equation}
j_1+j_2 = f \cdot ( u \cdot (v_1 \cdots v_n)_x +  u_x \cdot (v_1 \cdots v_n ) ).
\end{equation}
Therefore, $|J_3|=2,|J_4|=n+1\ge 2$. Such cases can be handled by the same procedure as the normal case, so they can be grouped together under the condition $\min(|J_3|,|J_4|)=2$. Such cases can be processed by Algorithm \ref{FindRuleForVars2}.

For the special case, there might be 
\begin{equation}
\begin{split}
j_1 + j_2   &= f \cdot ( j_3 g + j_4 g ) \\ 
            &= f \cdot ( (g \cdot t_1\cdots t_n)_x\cdot g + (g \cdot t_1\cdots t_n)\cdot g_x ),
\end{split}
\end{equation}
where $j_3=(g \cdot t_1\cdots t_n)_x , j_4=t_1\cdots t_n\cdot g_x$ with $\min(|J_3|,|J_4|)=1$. In such case, we have $TP(j_3)=complex, g\in FC(j_3)\cap F$. So we can try every possible $g$ to find combination rules as described in Algorithm \ref{FindRuleForVars1}. 

Finally, there is an external parameter $S$ in the pseudo-code. It means that there are some external integral operations outside the current recursion. As we know, if there is no integral operation, such as $u_x v + u v_x = (uv)_x$, a combination would be restored after expansion. By default, we only combine the integrals like $\int\!{u_x v\dd x} + \int\!{u v_x\dd x} = uv$ by taking $S=\varnothing$. But if we want to perform the previous combination, we can set $S$ to be the whole variable set $\mathcal V$. 
% As described in \refsec{implementation-03}, this behavior is controlled by the parameter \texttt{CombineDiff}.

In the Algorithms \ref{FindRuleForVars2} and \ref{FindRuleForVars1}, we only search the differential rules about variables in the set $S$.  The basic rule $u_x v + u v_x = (uv)_x$ holds when $x$ is the independent variable of $u$ and $v$. So, we would reduce $S$ to the intersection of $FV(I_1),FV(I_2)$ and $S$, which is shown in the pseudo-code.

In short, we have explained the details of finding combination rules between two integral items. Here is a typical example for this algorithm. Let 
\begin{equation}
a=\int\!{uu_xv \dd x},~~b=\int\!{u^2v_x\dd x},~~c=\int\!{u(uv)_x\dd x},~~d=u^2v.
\end{equation}
We have the linear combination rules $\{a+b=c,a+c=d\}$. 

For $a+b=c$, we have $F=[u],J_3=[u_x,v],J_4=[u,v_x]$, which can be handled by the Algorithm \ref{FindRuleForVars2}.

For $a+c=d$, we have $F=[u],J_3=[(uv)_x],J_4=[u_x,v]$, which can be processed by the Algorithm \ref{FindRuleForVars1}.  

For the \emph{incomplete matched} expression $3a+b$,
we can get the optimized result $a+d$ by solving the corresponding optimization problem.

For polynomial $5a^2+4ab+b^2$, which is \emph{linear unreducible}, we would get the following 8 rules with 10 SIIs by the \emph{recursive cases} based on the Algorithm \ref{FindAllRules}, i.e., 
\begin{equation}
\left\{ 
\begin{matrix}
a^2+ab=ac, &ab+b^2=bc, &ac+bc=c^2, \\
ad+bd=cd,  &a^2+ac=ad, &ab+bc=bd,  \\ 
ac+c^2=cd, &ad+cd=d^2. &
\end{matrix}
\right\}
\end{equation}
As an example of these rules, for $a^2+ab=ac$, we would have $F=[a],J_3=[a],J_4=[b]$ for the first level of recursion, then get $a+b=c$ in the second level of recursion. Finally, we can get the best result $a^2+d^2$ by solving the related optimization problem. 

\subsection{Discussion of optimization methods}\label{optMethods-03}

As described in \refsec{optimization-03}, the simplification problem is equivalent to a linear programming (LP) problem. But, our problem cannot be solved by numerical LP algorithms directly. Once existing numerical errors, there would be a lot of superfluous rest items with tiny coefficients. It is difficult to determine whether these items actually exist or caused by numerical errors. Thus, we need to do exact linear programming (ELP) in our problem. 

There are some ELP implementations, such as \texttt{simplex} in Maple, \texttt{LinearProgramming} in Mathematica, \spell{SoPlex} \citep{soplex} and \spell{Qsopt-ex} \citep{qsoptex}. Since most ELP implementations aim to solve the problem on rational field, we should transform our problem into it. 

Review the definition of SIE in \refeqn{std_form}, we assume the coefficients $c_k\in K$, and the SIIs $I_k\in \mathbb F$. The simplification problem is dependent on these two sets, we use $K[\mathbb F]$ to denote it. In general, we should solve $\mathbb C[\mathbb F]$. Regard the imaginary unit $i$ as a constant, $\mathbb C[\mathbb F]=\mathbb R[\mathbb F \cup \{i\}]$. Furthermore, regard all  irrational numbers as constants, we have $\mathbb C[\mathbb F]=\mathbb Q[F\cup (\mathbb R - \mathbb Q) \cup \{i\}]$. Finally, we can solve the ELP on rational field.

For example, consider
\begin{equation}
\left(1+\sqrt{2}\right)\ii{u_x v}+\sqrt{2}\ii{u v_x}, 
\end{equation}
we have
\begin{equation}
\bm{B}=\left(
\begin{array}{c}
\ii{u_x v}  \\
\ii{u v_x}  \\
u v         \\
\sqrt{2}\ii{u_x v}  \\
\sqrt{2}\ii{u v_x}  \\
\sqrt{2}u v         \\
\end{array}
\right)\TT,
\bm{b}=\left(
\begin{array}{c}
1   \\
0   \\
0   \\
1   \\
1   \\
0   \\
\end{array}
\right),
\bm{A}=\left(
\begin{array}{rr}
1   & 0\\
1   & 0\\
-1  & 0\\
0   & 1\\
0   & 1\\
0   & -1\\
\end{array}
\right).
\end{equation}
Solve the related ELP, the optimal solution is $\bm{x}=\left(0,1\right)\TT$, and the simplified result is $\sqrt{2}uv+\ii{u_x v}$. 

There is another way to avoid numerical errors. We can do numerical integer linear programming (NILP) instead of ELP, and round the result to integers. In theory, ILP is more difficult than LP, because ILP is NP-hard. But, numerical algorithms is faster than symbolic algorithms in practice. So, NILP algorithms maybe faster than ELP algorithms in some cases.

Based on the considerations above, we have integrated several optimization algorithms in our program:
\begin{compactitem}[\textbullet]
\item \texttt{SLP}, which means \texttt{simplex} in Maple, is a ELP algorithm; 
\item \texttt{mmaSLP}, which means \texttt{LinearProgramming} in Mathematica, is a ELP algorithm;
\item \texttt{SoPlex} \citep{soplex} is a ELP algorithm that implemented in C++;
\item \texttt{NILP}, which means \texttt{LPSolve} in Maple, is a NILP algorithm;
\item \texttt{MatlabILP}, which means \texttt{intlinprog} in Matlab, is a NILP algorithm.
\end{compactitem}

Now, a concrete example is given to compare their running time for different scales. Let $\mathcal F=\{f_1,f_2,\cdots\}$, consider the expression
\begin{equation}
\left(\sum\limits_{k=1}^{n_1}{\int\!{(f_{i_{2k-1}}f_{i_{2k}})_x\dd x}}\right)^2+\sum\limits_{k=1}^{n_2}{\int\!{(f_{j_{3k-2}}f_{j_{3k-1}}f_{j_{3k}})_x\dd x}},
\end{equation}
where $i_k,j_k \in \{1,2,\cdots\}$. In order to control the number of generated items, we take $n_1\le 8,n_2\le 8$ in the experiment. The runtime of solving optimization problems by different algorithms is shown in \reffig{opts_log}. For a clearer comparison of them, we take the logarithm of runtime.

\begin{figure}[htb]
\centering
\includegraphics[width=0.8\textwidth]{fig/int-6.pdf}
\caption{Log scaled runtime about optimization methods}\label{opts_log}
\label{opts_all}
\end{figure}

It can be seen from \reffig{opts_all}, the growth trend of runtime of ELP algorithms are similar. But \texttt{SoPlex} is much faster than the others. The runtime of \texttt{SoPlex} did not exceed 2 seconds in our tests.

Although ILP is much harder than LP, the advantage of numerical calculation makes numerical ILP algorithms faster than symbolic LP algorithms. For example, \texttt{NILP} is faster than \texttt{SLP} and \texttt{mmaSLP} when the scale of problem is less than 250. The further developed NILP algorithm \texttt{MatlabILP} is significantly faster than \texttt{SLP} and \texttt{mmaSLP} within our experiments. 

In addition, we can't say that \texttt{MatlabILP} is faster than \texttt{SoPlex}. Because we invoke \texttt{SoPlex} through command line interface, which spend much time in file IO and program launching.

In conclusion, \texttt{SoPlex} and \texttt{MatlabILP} are the most efficient optimization algorithms in our problem.

\section{Experiments} \label{Results-03}
In this section, we will compare our algorithm with those existing ones in \refsec{sec5.1-03}, and analyze the time complexity of our algorithm in \refsec{sec5.2-03}.

\subsection{Comparison of capability of simplification}\label{sec5.1-03}
In order to compare our algorithm with those existing ones in Maple and Mathematica in detail,  we introduce the following 7 criterions to classify integrals:
\begin{compactenum}[1) ]
\item The number of multiplied functions inside an integral is more than 2.
\item The expression contains incomplete matched integrals.
\item Some nested integrals should be regarded as differentials.
\item The nested integrals can be simplified recursively. 
\item The polynomial of integrals is linear unreducible.
\item The abstract function is composite with nonlinear functions, such as $u^{-1}$, $\sin(u)$, etc.  
\item The expression contains specific functions such as $x$, $\sin(x)$, etc. 
\end{compactenum} 
These 7 criterions divide all integrals into 128 classes. The classification of an integral can be represented by a binary sequence with the length 7, where 1 represents true and 0 represents false. The increase in the number of 1 in a binary sequence means a more general situation.

As shown in Table \ref{tb1}, based on the control variate method, we choose 10 different classes of integral to show the effectiveness of each criterion. For each case, we construct a typical integral for testing. Limited by the table width, we use a simplified notation to represent integrals. For example, $\int\!{(uv)_x\dd x}=\int\!{u_xv\dd x}+\int\!{uv_x\dd x}$. The differential inside an integral will be expanded, not eliminated. The first column indicates the classification of the expression. The binary sequence in the second column means the capability of Maple, Mathematica and our algorithm successively. The testing code of Maple is \texttt{eval(factor(combine(e)))}. The testing code of Mathematica is 
\begin{verbatim}
Factor[e//.{
    p_?NumericQ*Integrate[a_,c_]+Integrate[b_,c_]
    :>Integrate[p*a+b,c],
    Integrate[a_,c_]+Integrate[b_,c_]
    :>Integrate[a+b,c],
    p_?NumericQ*Integrate[a_,c_]
    +q_?NumericQ*Integrate[b_,c_]
    :>Integrate[p*a+q*b,c]
}];
\end{verbatim}

% 1. 二项、多项
% 2. 完全匹配,不完全匹配
% 3. 无积分,积分作为导数
% 4. 无递归积分,有递归积分
% 5. 是否线性可约
% 6. 有无复合
% 7. 有无混合
\begin{table}[htb]
\renewcommand{\arraystretch}{1.25}
\centering
\caption{Comparison of capability of simplification} \label{tb1}
\renewcommand{\dd}[1]{\mathrm{d}#1}
\renewcommand{\ii}[1]{\int\!{#1\dd x}}
\begin{tabular}{ccl}
\hline
Type & Result & \multicolumn{1}{c}{Expression} \\
\hline
0000000 & 111 & $I_1=\int\!{(uv)_x\dd x}$\\ 
1000000 & 111 & $I_2=\int\!{(u^2v)_x\dd x}$\\ 
0100000 & 001 & $I_3=I_1+\int\!{u_xv\dd x}$\\ 
0010000 & 001 & $I_4=\int\!{(\int\!{u\dd x}\cdot \int\!{v\dd x})_x\dd x}$\\
0001000 & 001 & $I_5=\int\!{u\cdot \int\!{(vw)_x\dd x}\dd x}$\\
0000100 & 001 & $I_6=I_1^2+\int\!{(vw)_x\dd x}$\\
0000010 & 100 & $I_7=\int\!{(\sin(u)\sqrt{v})_x\dd x}$\\
0000001 & 110 & $I_8=\int\!{(\sin(x)u)_x\dd x}$\\
1111100 & 001 & $I_9=\int\!{(\int\!{u\dd x}\int\!{(\int\!{uv\dd x}\;w)_x\dd x\dd x})_x\dd x}+I_3^2$\\
1111111 & 000 & $I_{10}=I_9+\int\!{(\sin(x)/u)_x\dd x}$\\
\hline
\end{tabular}
\end{table} 

As we can see from Table \ref{tb1}, Maple and Mathematica can only process linear reducible polynomials of simple integrals. They require that the inside of integrals should be complete matched and do not contain nested integrals. In addition, Mathematica may fail to simplify some complete matched integrals. However, our proposed algorithm \texttt{IntCombine} can handle arbitrary polynomials of complex integrals. The inside of an integral can be incomplete matched and even contain nested integrals. From $I_7$ and $I_8$ we know, \texttt{IntCombine} do not support expressions containing specific or composite functions, but Maple and Mathematica can process simple cases of them. However,  \texttt{IntCombine} can process all other cases according to $I_9$. The most general case $I_{10}$ cannot be handled by any algorithm mentioned in Table \ref{tb1}. 

To further show the effectiveness of \texttt{IntCombine}, a more complicated example that contains 50 items is considered, i.e.,
\begin{equation}
\def\arraystretch{1.5}
\begin{array}{l}
-20u_xw_{xx}v
+c\ii{u_{xxxxx}\ii{v}}
+c\ii{u_{xxxxx}v}
+c\ii{v_{xxxxx}u}\\% 1
+20u_x^2v_xa
+20u_x\ii{wv_{xxx}}
+20w_x\ii{v_xu_{xx}}
+20w_x\ii{u_xv_{xx}}\\% 2
+10c\ii{v_{xxx}u^2}
+30c\ii{uu_xv_{xx}}
-20u_xwv_{xx}
-60cuu_x\ii{uv}\\% 3
+10c\ii{u_x\ii{u_xv_{xx}}}
+10c\ii{u_x\ii{v_xu_{xx}}}
-30vu^3c
-40u_xw_xv_x\\% 4
+30c\ii{u^2u_{xx}\ii{v}}
+20c\ii{v_xu_{xx}u}
+30c\ii{u_{xxx}u_x\ii{v}}\\% 5
+30c\ii{u_{xx}vu_x}
+40u_x\ii{w_xv_{xx}}
+40u_x\ii{v_xw_{xx}}
-120uu_xwv\\% 6
-4u_x^2v_xb
+20u_x\ii{vw_{xxx}}
+120u_x\ii{vuw_x}
+120u_x\ii{uvw_x}\\% 7
+60c\ii{uu_x^2\ii{v}}
+120c\ii{u^2u_xv}
-cuv_{xxxx}
-10u_{xxx}c\ii{vu}\\% 8
-10u_{xxx}c\ii{u_x\ii{v}}
-20u_xa\ii{u_xv_{xx}}
-20u_xa\ii{v_xu_{xx}}\\% 9
+4u_xb\ii{u_xv_{xx}}
-10cu\ii{u_xv_{xx}}
-10c v u u_{xx}
-10cu\ii{v_xu_{xx}}\\% 10
+4u_xb\ii{v_xu_xx}
+20c\ii{u_{xx}^2\ii{v}}
-cu_{xxxxx}\ii{v}\\% 11
+10c\ii{u_{xxxx}u\ii{v}}
+30u^2u_xc\ii{v}
-60uu_xc\ii{u_x\ii{v}}\\% 12
-20cu_xu_{xx}\ii{v}
+120u_x\ii{vwu_x}
+c\ii{u_xv_{xxxx}}
-10v_{xx}u^2c\\% 13
+30c\ii{u^3v_x}
+20c\ii{vu_{xxx}u},\\% 14
\end{array}
\label{big50}
\end{equation}
where $u=u(x,t),v=v(x,t),w=w(x,t)$, and $a,b,c$ are constants. 

Our algorithm can simplify it within 4 seconds, the obtained result is 0. There are a lot of complicated combination rules used in this example. Due to the space limitations, we only show two of them here. Firstly, we have 
\begin{equation}
\def\arraystretch{1.5}
\begin{array}{l}
0=u_x\int\!{v_{xx}w_x\dd x}+u_x\int\!{v_xw_{xx}\dd x}-u_xv_xw_x\\
~~+w_x\int\!{u_xv_{xx}\dd x}+w_x\int\!{u_{xx}v_x\dd x}-u_xv_xw_x\\
~~+u_x\int\!{vw_{xxx}\dd x}+u_x\int\!{v_xw_{xx}\dd x}-u_xvw_{xx}\\
~~+u_x\int\!{v_{xxx}w\dd x}+u_x\int\!{v_{xx}w_x\dd x}-u_xv_{xx}w.
\end{array}
\label{counter_example}
\end{equation}
Here, $u_x\int\!{v_{xx}w_x\dd x},u_x\int\!{v_xw_{xx}\dd x}$ and $u_xv_xw_x$ are appropriately distributed to four different combinations, which needs some human intelligence but can be done automatically by \texttt{IntCombine}. It is a good example to show why we cost so much time to find all combination rules rather than to simplify the expression step by step. Because the rest six items will not be combined if we eliminate these three items first. 

Secondly, we have 
\begin{equation}
\def\arraystretch{1.5}
\begin{array}{l}
c\int\!{\int\!{v\dd x}uu_{xxxx}\dd x}+c\int\!{\int\!{v\dd x}u_xu_{xxx}\dd x}=c\int\!{\int\!{v\dd x}(uu_{xxx})_x\dd x},\\
c\int\!{\int\!{v\dd x}(uu_{xxx})_x\dd x}+c\int\!{vuu_{xxx}\dd x}=cuu_{xxx}\int\!{v\dd x}.
\end{array}
\end{equation} 
The diversity of nested integrals like this shows another advantage of our algorithm. 

\subsection{Analyzing the time complexity}\label{sec5.2-03}
In this section, we will analyze the time complexity of our algorithm in practice.

In order to test time cost of the worst cases in practice, we take $n$ in \refeqn{worst_case} from 2 to 8, and get the results as shown in \reffig{order}. It can be seen that the time cost grows exponentially. Roughly, the runtime times 4 as the order increases 1, which is consistent with the theoretical analysis in \refsec{all_rules-03}. 

\begin{figure}[htb]
\centering
\subfigure{
    \includegraphics[width=0.45\textwidth]{fig/int-1.pdf}
    \label{order}
}
\subfigure{
    \includegraphics[width=0.45\textwidth]{fig/int-2.pdf}
    \label{degree}
}
\caption{(a) Runtime about order; (b) Runtime about degree}
\end{figure}

It seems that our algorithm is terrible in complexity. However, in practice, there are few cases that more than eight different functions are multiplied in an expression.

In fact, consider the polynomial 
\begin{equation}
\left(\int\!{(f_1\cdots f_n)_x\dd x}\right)^m,
\label{poly}
\end{equation}
where $f_1,\cdots,f_n$ are different functions. The inner part has $M=2^n-1$ items in our algorithm. The number of  monomials generated by them is $\binom{m+M-1}{m}$, where $\binom{A}{B}$ represents binomial coefficient. Thus, the complexity is 
\begin{equation}
\mathcal O\left(\binom{m+2^n-2}{m}^2\right).
\label{polynomial_complexity}    
\end{equation}
For large $n$, it is still exponential. However, for limited $n$, it is equivalent to $\mathcal O(m^{2N})$, where $N=2^n-2$. In other words, the complexity is polynomial, which is acceptable in practice. 

We take $n=2,m\le 20$ in \refeqn{poly}, the runtime of these expressions are shown in \reffig{degree}. For polynomial whose degree is more than 20, our program can solve it within 8 seconds. 

Our algorithm will be more efficient for lower order and lower degree expressions, even for expressions with a large number of input items. For such inputs, there are two techniques to  accelerate the calculation.

The first technique is to keep a remember table. In general, different integral items might contain the same components, so the combine rules of \emph{root cases} might be the same. For example, as mentioned in the end of \refsec{Combine-03}, rules like $ad+bd=cd$ and $ac+bc=c^2$ will be reduced to the same root rule $a+b=c$. It can be accelerated if we remember all rules have been used. It can be implemented easily in Maple by taking advantage of the \verb|option remember| \citep{maple_programming}. 

The second technique is to group the input expression. We notice that two SIIs only can be combined if they have the same $FR$, where
\begin{equation}    
FR(f)=\left\{
\begin{array}{cl}
\prod\limits_{g\in FC(f)}{FR(g)}, &\text{if }TP(f)=complex,\\ 
f,           &\text{if }TP(f)=simple.
\end{array}
\right.
\end{equation} 
Thus, we can use this to accelerate calculation. 

According to \refeqn{polynomial_complexity}, for limited orders and degrees, the time cost of a group would be a limited constant, so the complexity would be linear about the number of input items.  

Let $\mathcal F=\{f_1,f_2,\cdots,f_n\}$, consider the expression
\begin{equation}
\left(\sum\limits_{k=1}^{n_1}{\int\!{(f_{i_{2k-1}}f_{i_{2k}})_x\dd x}}\right)^4+\left(\sum\limits_{k=1}^{n_2}{\int\!{(f_{j_{3k-2}}f_{j_{3k-1}}f_{j_{3k}})_x\dd x}}\right)^2,
\label{construct}
\end{equation}
where $i_k,j_k \in \{1,2,\cdots,n\}$. The degree of $I$ is 4, and the order is no more than 8. In order to limit the number of input items, we take $n=8,n_1\le 5,n_2\le 5$, and using \texttt{MatlabILP} to do optimization, the runtime of these inputs is shown in \reffig{items_all}. 

\begin{figure}[htb]
\centering
\subfigure{
    \includegraphics[width=0.45\textwidth]{fig/int-3.pdf}
    \label{items_input}
}
\subfigure{
    \includegraphics[width=0.45\textwidth]{fig/int-4.pdf}
    \label{items_gen}
}
\caption{(a) Runtime about number of input items; (b) Runtime about number of generated items}
\label{items_all}
\end{figure}

As shown in \reffig{items_input}, the increasement of time is approximately linear about the number of input items. The runtime  in \reffig{items_gen} is more likely  a linear function about the number of generated items. The runtime sometimes decreases due to the remember strategy.

In conclusion, the complexity of our algorithm is exponential about the order of input expression in the worst cases. For the expression with limited orders, the complexity is polynomial about its degree. The complexity is linear when both order and degree are limited. In practice, \texttt{IntCombine} is usually efficient. 

\section{Conclusion} \label{Conclusion-03} 
We have designed a simplification algorithm for arbitrary polynomials of (nested) integrals that might be incomplete matched. The algorithm can handle many cases that cannot be processed by Maple and Mathematica. Furthermore, experiments show that our algorithm is efficient in practice.
