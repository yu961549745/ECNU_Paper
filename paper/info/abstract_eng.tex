% Abstract
\phantomsection
\addcontentsline{toc}{chapter}{Abstract}
\centerline{\zihao{3}\bfseries ABSTRACT}

\linespread{1.4}\zihao{-4}
\bigskip

Nonlinear systems are the widespread mathematical models for natural phenomena. The research of them plays an important role in many fields. The exact solutions of nonlinear systems is often accompanied by a large number of complex symbolic reasoning and calculations. With the development of high-performance computers and computer algebra systems, symbolic computation has become a powerful tool for solving problems related to nonlinear systems. Based on the symbolic computing platform Maple, this paper studies the related automatic algorithms for constructing exact solutions of nonlinear systems. Our work consists of three parts.

The first part is focuses on the solving of exact solutions to nonlinear systems.

The Hirota method is an effective method for solving nonlinear differential equations. Based on the Hirota method, we can construct soliton, breather, lump and rogue wave solutions. However, the $n$-soliton solution derived by the Hirota method  only works for integrable equations, and often does not hold for non-integrable equations. This paper introduce parameter constraints to extend the Hirota method to  non-integrable equations. Based on the simplified Hirota method, conjugate parameter method and long limit method, an automatic algorithm for constructing soliton, breather and lump solutions of the nonlinear evolution equations is developed, and the corresponding software TwSolver is developed. The software is applicable to both integrable and non-integrable equations with friendly interfaces. 

Direct algebraic methods are also widely used when solving differential equations. Its main difficulty is the solving of large-scale nonlinear algebraic equations. In order to address this problem, we designed a parallel grouped solve algorithm and develope a corresponding software PGSolve. As an application of PGSolve, we develope  the NS1L,  which used to construct $n$-soliton and 1-lump interaction solutions. Experiments shows that PGSolve can solve equations with a size less than 350 in a few minutes; for equations with a size above 100, its efficiency is more than 100 times higher than the ordinary solving function.

The second part mainly studies on the $n$-order expansion method and its application.

In the solution of differential equations, many methods are developed based on the  homogeneous balance principle, such as Painlevé expansion method, tanh method and Jacobi elliptic function method. These methods usually assume that the solution of the equation is a polynomial of a certain function, than determine the order of the solution by balancing the order of two different highest terms. However, the order of the solution cannot be determined when the highest terms have the same order, which is the main reason of loss solutions. We found that the order of the solution will not only appears in the order of the highest $n$ term, but also in the coefficient of the highest $n$ term. Therefore, we proposed the $n$-order expansion method, which would balance both order and coefficients of the highest $n$ terms. The corresponding software is named as NEM.  Based on NEM, a new algorithm for solving the polynomial solutions of the nonlinear difference equation is proposed, and the corresponding software NLREP is developed. At the same time, this paper also apply the $n$-order expansion method to tanh method and Painlevé expansion method. Experiments and examples show that the $n$-order expansion method theoretically perfected the homogeneous balance principle, and can derive more solutions.

The third part mainly studies on the simplification of nonlinear integral expression about abstract functions.

Because the simplification of integral expression is often needed in the process of solving nonlinear differential equations, this paper studies nonlinear integral simplification. Firstly, we establishe an algebraic system which  regard the integral polynomial as a linear combination of standard integral terms. Then, based on the derivative multiplication rule, a recursive algorithm is designed to find all binomial merge rules. Finally, based on these rules, the simplification problem is transformed into an exact linear programming problem. The corresponding software  IntSimplify is implemented to simplify the nonlinear integral expression. IntSimplify can simplify the integral polynomial with nested integrals and redundant terms, which can simplify more complex integral expressions compared with existing algorithms.

\bigskip
\noindent\textbf{\zihao{4} Keywords:}
nonlinear system, exact solution, differential equation, difference equation, integral simplification, Painlevé expansion method, Hirota method, conjugate parameter method, long wave limit method, parallel grouped solve algorithm, homogeneous balance principle, $n$-order expansion method, exact linear programming
