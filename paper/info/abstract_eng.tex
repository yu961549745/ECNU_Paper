% Abstract
\phantomsection
\addcontentsline{toc}{chapter}{Abstract}
\centerline{\zihao{3}\bfseries ABSTRACT}

\linespread{1.4}\zihao{-4}
\bigskip

Nonlinear systems are the main mathematical models for describing natural phenomena, and their research plays an important role in many fields. Related research on the exact solutions of nonlinear systems is often accompanied by a large number of complex symbolic inferences and calculations. In recent years, with the development of high-performance computers and computer algebra systems, symbolic computing has become a powerful tool for solving problems related to nonlinear systems. Based on the symbolic computing platform Maple, this paper studies the related mechanization algorithms for constructing exact solutions of nonlinear systems, and carries out the following three aspects.

The first part of the research work on the exact solution of the nonlinear system.

The Hirota method is an effective method for solving nonlinear differential equations. Based on the Hirota method, the soliton solution, the breather solution, the lump solution and the strange wave solution of the nonlinear differential system can be constructed. However, the n-soliton solution formula derived by the Hirota method only holds for the integrable equation, and often does not hold for the integrable equation. By introducing parameter constraints to extend the Hirota method to the integrable equation, and based on the simple Hirota method, the conjugate parameter method and the long limit method, the soliton solution, the respiratory solution and the mass of the nonlinear evolution equation are developed. The mechanized algorithm of the solution was developed and the corresponding symbol calculation software TwSolver was developed. The software is suitable for both integrable and integrable equations and has a friendly interface.

The direct algebraic method is also widely used in the solution of differential equations. Its main difficulty is the solution of large-scale nonlinear algebraic equations. Aiming at the difficult problem of solving large-scale nonlinear algebraic equations, this paper designs a grouping parallel algorithm and implements it as PGSolve software package. As an application example of PGSolve, this paper develops a software package NS1L that uses the direct algebraic method to find the n-soliton and 1-lump interaction solutions. Based on the large-scale equations generated by NS1L, we found that PGSolve can solve equations with a scale of less than 350 in a few minutes; for equations with a size above 100, the solution efficiency is more than 100 times higher than the ordinary solution function.

The second part mainly studies the n-order expansion method and its application.

In the solution of differential equations, many methods are developed based on the principle of homogeneous balance, such as Painlevé analysis, hyperbolic tangent method and Jacobi elliptic function method. These methods usually set the solution of the equation to a polynomial of a function, and determine the order of the solution by balancing the order of two different highest terms in the equation. However, when the expressions of the order of the highest order of an equation are the same, the upper bound of the order of the solution cannot be determined, and thus it is possible to miss. This paper finds that the order of the solution will not only appear in the order of the highest term, but also in the coefficient of the non-highest term. Therefore, this paper considers the order and coefficient of the highest n terms in the equilibrium equation at the same time, and proposes an n-th order expansion method to implement the NEM software package. Based on the n-th order expansion method, a new algorithm for solving the nonlinear difference equation polynomial solution is proposed and implemented as the NLRREP software package. At the same time, this paper also applies the n-order expansion method to hyperbolic tangent method and Painlevé analysis. Experiments and examples show that the n-th order expansion method is perfected by theoretically aligning the sub-equilibrium principle, which can analyze the equilibrium situation more comprehensively and obtain more solutions when solving.

The third part mainly studies the simplification of the nonlinear integral expression of the abstract function.

Because the simplification of integral expression is often needed in the process of solving nonlinear differential equations, this paper takes nonlinear integration into a challenging problem. First, this paper establishes an algebraic system to treat the integral polynomial of the abstract function as a linear combination of standard integral terms. Then, based on the derivative multiplication rule, a recursive algorithm is designed to find all binomial merge rules. Finally, based on these rules, the simplification problem is transformed into a precise linear programming problem, and the software package IntSimplify.IntSimplify, which implements the nonlinear integral expression simplification, can simplify the integral polynomial with nested integrals and redundant terms. More complex integral expressions can be reduced than existing algorithms.

\bigskip
\noindent\textbf{\zihao{4} Keywords:}
nonlinear system, exact solution, differential equation, difference equation, integral simplification, Painlevé expansion method, Hirota method, conjugate parameter method, long wave limit method, parallel grouped solve algorithm, homogeneous balance principle, n-order expansion method, exact linear programming
