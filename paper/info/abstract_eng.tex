% Abstract
\phantomsection
\addcontentsline{toc}{chapter}{Abstract}
\centerline{\zihao{3}\bfseries ABSTRACT}

\linespread{1.4}\zihao{-4}
\bigskip
Nonlinear systems are the main mathematical models for describing natural phenomena, and their research plays an important role in many fields. The related research on the exact solutions of nonlinear systems is often accompanied by a large number of complex symbolic reasoning and calculations. With the development of high-performance computers and computer algebra systems, symbolic computation has become a powerful tool for solving problems related to nonlinear systems. Based on the symbolic computing platform Maple, this paper studies the related mechanization algorithms for the exact solution of nonlinear systems. Aspect work.

The first part is developed around the solution of nonlinear differential equations.

In the solution of differential equations, the Hirota method has a wide range of applications. Based on the Hirota method, different types of solutions such as soliton solutions, breather solutions, lump solutions and strange wave solutions can be obtained. However, the Hirota method can only guarantee the integrable equations. True solution. This paper introduces the parameter constraints to extend the Hirota method to the integrable equation, and implements a software package TwSolver which can solve the three wave solutions of the nonlinear evolution equation. In TwSolver, this paper first uses the Painlevé analysis to determine the equation. Transform, and then get the correct soliton solution based on the simple Hirota method and the appropriate parameter constraints. After that, the breather solution is obtained by the conjugate parameter method, and the lump solution is obtained by the long limit method. This paper not only integrates the above methods, but also The programming implementation has been optimized. Experiments and examples show that TwSolver can get the true solution whether it is an integrable equation or an integrable equation.

The direct algebraic method is also widely used in the solution of differential equations. Its main difficulty is the solution of large-scale algebraic equations. In order to solve the difficult problem of large-scale algebraic equations, this paper designs a group-parallel algorithm and It is implemented as the PGSolve software package. As an application example of PGSolve, this paper develops a software package NS1L that uses the direct algebraic method to find the n-soliton and 1-lump interaction solutions. Based on the large-scale equations generated by NS1L, we found that: PGSolve can solve equations with a size of less than 350 in a few minutes; for equations with a size above 100, the solution efficiency is more than 100 times higher than the ordinary solution function.

The second part mainly studies the nth order expansion method and its application.

In the solution of differential equations, many methods are based on the principle of homogeneous balance, such as Painlevé analysis, hyperbolic tangent method and Jacobi elliptic function method. The homogeneous balance principle sets the solution of the equation as a polynomial of a function. The order of the solution is determined by balancing the order of two different highest terms in the equation. However, when the two highest terms of the same order are balanced, the homogeneous balance principle cannot determine the upper bound of the order, which may result in The case of leakage. It is found that the order of the solution will not only appear in the order of the highest term, but also in the coefficient of the non-highest term. Therefore, this paper considers the order and coefficient of the highest n term in the simultaneous equilibrium equation. An n-th order expansion method is proposed to implement the NEM software package. Based on the n-th order expansion method, we propose a new algorithm for solving the nonlinear difference equation polynomial solution and implement it as the NLUPS software package. At the same time, this paper will also The order expansion method is applied to hyperbolic tangent method and Painlevé analysis. Experiments and examples show that the n-th order expansion method is perfected from the theoretically aligned sub-equilibrium principle and can be more comprehensively Case analysis of balance, get more solutions in solving.

In the third part, we mainly study the simplification of nonlinear integral expressions.

Because it is often necessary to simplify the integral expression in the process of solving nonlinear differential equations, this paper studies nonlinear integral simplification as a challenging task. First, this paper establishes an algebraic system that will be about abstract functions. The integral polynomial is regarded as a linear combination of standard integral terms. Then, based on the derivative multiplication rule, a recursive algorithm is designed to find all binomial merge rules. Finally, based on these rules, the reduction problem is transformed into an exact linear programming problem. Solving, the software package IntSimplify is implemented to simplify the nonlinear integral expression. IntSimplify can simplify the integral polynomial with nested integrals and redundant terms, and can simplify more complex integral expressions compared with existing algorithms.

\bigskip
\noindent\textbf{\zihao{4} Keywords:}
nonlinear system, exact solution, differential equation, difference equation, integral simplification, Painlevé analysis, Hirota method, conjugate parameter method, long wave limit method, parallel grouped solve algorithm, homogeneous balance principle, n-order expansion method, exact linear programming
