% Abstract
\phantomsection
\addcontentsline{toc}{chapter}{Abstract}
\centerline{\zihao{3}\bfseries ABSTRACT}

\linespread{1.4}\zihao{-4}
\bigskip

Nonlinear systems are the widespread mathematical models for natural phenomena. The research of them plays an important role in many fields. The exact solutions of nonlinear systems is often accompanied by a large number of complex symbolic reasoning and calculations. With the development of high-performance computers and computer algebra systems, symbolic computation has become a powerful tool for solving problems related to nonlinear systems. Based on the symbolic computing platform Maple, this paper studies the related automatic algorithms for constructing exact solutions of nonlinear systems. Our work consists of three parts.

The first part is the exact solution of the nonlinear system.

The Hirota method is an effective method for solving nonlinear differential equations. Based on the Hirota method, we can construct soliton, breather, lump and rouge wave solutions. However, the n-soliton solution derived by the Hirota method  only works for integrable equations, and often does not hold for non-integrable equations. This paper introduces the parameter constraints to extend the Hirota method to the integrable equation, and develops based on the simple Hirota method, the conjugate parameter method and the long limit method. The automatic algorithm for constructing the soliton solution, the breather solution and the lump solution of the nonlinear evolution equation is developed, and the corresponding symbol calculation software TwSolver is developed. The software is applicable to both integrable and integrable equations, and has a friendly interface.

The direct algebraic method is also widely used in the solution of differential equations. Its main difficulty is the solution of large-scale nonlinear algebraic equations. In order to solve the difficult problem of large-scale nonlinear algebraic equations, this paper designs a group-parallel Solve the algorithm and develop the corresponding software PGSolve. As an application example of PGSolve, this paper develops the software package NS1L which uses n-soliton and 1-lump interaction solution by direct algebra method. Experiment based on large-scale equations generated by NS1L We found that: PGSolve can solve equations with a scale of less than 350 in a few minutes; for equations with a size above 100, the solution efficiency is more than 100 times higher than the ordinary solution function.

The second part mainly studies the n-order expansion method and its application.

In the solution of differential equations, many methods are developed based on the principle of homogeneous balance, such as Painlevé expansion method, hyperbolic tangent method and Jacobi elliptic function method. These methods usually set the solution of the equation to some The polynomial of a function, which determines the order of the solution by balancing the order of two different highest terms in the equation. However, when the expressions of the order of the highest order of an equation are the same, the order of the solution cannot be determined. It is possible to miss the solution. It is found that the order of the solution will not only appear in the order of the highest term, but also in the coefficient of the highest term. Therefore, this paper considers the order of the highest n term in the simultaneous equilibrium equation. And the coefficient, the n-th order expansion method is proposed, and the NEM software package is implemented. Based on the n-th order expansion method, a new algorithm for solving the polynomial solution of the nonlinear difference equation is proposed, and the corresponding software NNLREP is developed. At the same time, this paper will also The order expansion method is applied to the hyperbolic tangent method and the Painlevé expansion method. Experiments and examples show that the n-th order expansion method is perfected from the theoretically aligned sub-equilibrium principle and can be analyzed more comprehensively. Heng situation, the more solutions in solving.

The third part mainly studies the simplification of the nonlinear integral expression of the abstract function.

Because the simplification of integral expression is often needed in the process of solving nonlinear differential equations, this paper studies nonlinear integral simplification as a challenging problem. Firstly, this paper establishes an algebraic system that will be about abstract functions. The integral polynomial is regarded as a linear combination of standard integral terms. Then, based on the derivative multiplication rule, a recursive algorithm is designed to find all binomial merge rules. Finally, based on these rules, the reduction problem is transformed into an exact linear programming problem. Solving, the software package IntSimplify is implemented to simplify the nonlinear integral expression. IntSimplify can simplify the integral polynomial with nested integrals and redundant terms, which can simplify more complex integral expressions compared with existing algorithms.

\bigskip
\noindent\textbf{\zihao{4} Keywords:}
nonlinear system, exact solution, differential equation, difference equation, integral simplification, Painlevé expansion method, Hirota method, conjugate parameter method, long wave limit method, parallel grouped solve algorithm, homogeneous balance principle, n-order expansion method, exact linear programming
