%%% 中文摘要
\phantomsection
\addcontentsline{toc}{chapter}{摘要}
\centerline{\zihao{-3}\heiti 摘\quad 要}

\linespread{1.4}\zihao{-4} \bigskip

非线性系统是描述自然现象的主要数学模型, 其研究在众多领域内都发挥着重要的作用. 而非线性系统的相关研究往往伴随着大量复杂的符号推理和计算. 近年来, 随着高性能计算机和计算机代数系统的发展, 符号计算已经成为解决非线性系统相关问题的有力工具. 本文基于符号计算平台Maple, 针对非线性系统的精确解以及相关基础算法进行研究, 开展了以下三个方面的工作. 

第一部分围绕着非线性微分方程的求解进行展开. 

首先, 基于 \Painleve{} 分析和简单 Hirota 方法, 实现了非线性演化方程三种波解的求解. 其中, 孤子解通过简单 Hirota 方法得到. 获得孤子解后, 通过共轭参数法得到呼吸子解, 通过长极限方法得到 lump 解. 上述过程是对已有方法的整合, 本文的主要工作在于将上述方法推广到不可积方程并对编程实现进行优化. 本文实现的 TwSolver 软件包在求解过程中引入了参数约束条件, 能够获得原方程的真解.  

直接代数方法在微分方程求解中也有广泛的应用, 但其往往伴随着大规模代数方程组的求解. 针对大规模代数方程组求解困难的问题, 本文设计了一个分组并行的求解算法, 并将其实现为 PGSolve 软件包. 作为 PGSolve 的应用实例, 本文开发了用直接代数方法求$n$-孤子和1-lump相互作用解的软件包 NS1L. 实验表明, PGSolve 对于含100个左右方程的方程组能够在几分钟内完成求解, 对于含300个左右方程的方程组能够在一小时内实现大部分分支的求解. 

第二部分主要研究非线性差分方程的求解以及相关算法在微分方程求解中的推广. 

受到在微分方程求解中被广泛应用的齐次平衡原则的启发, 本文用其求解非线性差分方程的多项式解. 齐次平衡原则通过平衡方程中最高次项的次数来确定解的次数, 只能在一些情况下生效. 本文考虑同时平衡方程中最高$n$项的次数和系数, 提出了$n$阶展开方法来处理齐次平衡原则不能处理的情况, 并将其实现为一个便于拓展应用的软件包 NEM. 基于NEM, 实现了能够求解非线性差分方程所有多项式解的软件包 NLREPS. 

在微分方程的求解中, NEM 能够快速地完善基于齐次平衡原则的求解算法. 首先, 我们以双曲正切方法为例, 基于NEM重新实现了一个NTCM软件包. 该软件包在将双曲正切方法推广到$n+1$维的同时, 利用NEM在阶数分析上的优势, 求解了许多以往不能求解的方程. 同时, 我们也以 \Painleve{} 分析为例展示了 NEM 的作用. 

在第三部分, 因为在非线性微分方程求解的过程中往往需要进行非线性积分表达式的化简, 本文将非线性积分化简作为一个具有挑战性的任务进行研究. 首先, 本文建立了一个代数系统将关于抽象函数的积分多项式视为标准积分项的线性组合. 然后, 基于导数的乘法规则, 设计了一个递归算法来寻找所有的二项合并规则. 最后, 基于这些规则将化简问题转化为一个精确线性规划问题进行求解, 实现了非线性积分表达式化简的软件包 IntSimplify.  IntSimplify能够化简含有嵌套积分和冗余项的积分多项式, 相比于已有的算法能够化简更加复杂的积分表达式. 

\bigskip

\noindent{\zihao{4}\heiti 关键词:}
非线性系统, 精确解, 微分方程, 差分方程, 积分表达式化简, \Painleve{}分析, Hirota方法, 共轭参数法, 长极限方法, 分组并行求解算法, 齐次平衡原则, $n$阶展开方法, 双曲正切方法, 精确线性规划



