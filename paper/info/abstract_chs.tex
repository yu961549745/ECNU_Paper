\phantomsection
\addcontentsline{toc}{chapter}{摘要}
\centerline{\zihao{-3}\heiti 摘\quad 要}

\linespread{1.4}\zihao{-4} \bigskip

非线性系统是描述自然现象的主要数学模型, 其研究在众多领域内都发挥着重要的作用. 而关于非线性系统精确解的相关研究往往伴随着大量复杂的符号推理和计算. 近年来, 随着高性能计算机和计算机代数系统的发展, 符号计算已经成为解决非线性系统相关问题的有力工具. 本文基于符号计算平台Maple, 针对构造非线性系统精确解的相关机械化算法进行研究, 开展了以下三个方面的工作. 

第一部分围绕非线性系统的精确求解开展了以下研究工作. 

Hirota方法是求解非线性微分方程的一种有效方法. 基于 Hirota 方法可构造非线性微分系统的孤子解\D 呼吸子解\D lump 解和怪波解等. 但是, 由 Hirota 方法推导出的$n$孤子解公式只对可积方程成立, 对不可积方程往往不成立. 本文通过引入参数约束条件, 将 Hirota 方法推广到不可积方程的情形, 并基于简单Hirota 方法\D 共轭参数法和长极限法, 发展出了构造非线性发展方程的孤子解\D 呼吸子解及 lump 解的机械化算法, 并研发了相应的符号计算软件TwSolver. 该软件对可积方程和不可积方程都适用, 且有友好的使用接口. 

直接代数方法在微分方程的求解中也有广泛的应用, 它的主要难点是其中大规模非线性代数方程组的求解. 针对大规模非线性代数方程组求解困难的问题, 本文设计了一个分组并行的求解算法, 并将其实现为 PGSolve 软件包. 作为 PGSolve 的应用实例, 本文开发了用直接代数方法求$n$-孤子和1-lump相互作用解的软件包 NS1L. 基于 NS1L 产生的大规模方程组进行实验, 我们发现: PGSolve 能够在几分钟内求解规模在 350 以内的方程组; 对于规模在 100 以上的方程组, 其求解效率比普通求解函数高 100 倍以上. 

第二部分主要研究$n$阶展开方法及其应用. 

在微分方程的求解中, 有许多方法都是基于齐次平衡原则发展起来的, 如\Painleve{}分析\D 双曲正切方法和 Jacobi 椭圆函数方法等. 这些方法通常都是将方程的解设为某个函数的多项式, 通过平衡方程中两个不同最高项的阶数来确定解的阶数. 但是, 当一个方程的各个最高项的阶数的表达式相同时, 则无法确定解的阶数的上界, 从而有可能漏解. 本文发现, 解的阶数不仅会出现在最高项的阶数中, 还会出现在非最高项的系数中. 因此, 本文考虑同时平衡方程中最高$n$项的阶数和系数, 提出了$n$阶展开方法, 实现了 NEM 软件包. 基于$n$阶展开方法, 提出了一个求非线性差分方程多项式解的新算法, 并将其实现为 NLREPS 软件包. 同时, 本文还将$n$阶展开方法应用于双曲正切方法和\Painleve{}分析. 实验和例子表明, $n$阶展开方法从理论上对齐次平衡原则进行了完善, 能更加全面地分析平衡的情况, 在求解时获得更多的解. 

第三部分主要研究抽象函数的非线性积分表达式的化简.

因为在非线性微分方程的求解过程中往往需要进行积分表达式的化简, 本文将非线性积分化简作为一个具有挑战性的问题进行研究. 首先, 本文建立了一个代数系统将关于抽象函数的积分多项式视为标准积分项的线性组合. 然后, 基于导数的乘法规则, 设计了一个递归算法来寻找所有的二项合并规则. 最后, 基于这些规则将化简问题转化为一个精确线性规划问题进行求解, 实现了非线性积分表达式化简的软件包 IntSimplify.  IntSimplify能够化简含有嵌套积分和冗余项的积分多项式, 相比于已有的算法能够化简更加复杂的积分表达式. 

\bigskip

\noindent{\zihao{4}\heiti 关键词:}
非线性系统, 精确解, 微分方程, 差分方程, 积分表达式化简, \Painleve{}分析, Hirota方法, 共轭参数法, 长极限方法, 分组并行求解算法, 齐次平衡原则, $n$阶展开方法, 精确线性规划
