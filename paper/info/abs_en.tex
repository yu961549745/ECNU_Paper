% Abstract
\phantomsection
\addcontentsline{toc}{chapter}{Abstract}
\centerline{\zihao{3}\bfseries ABSTRACT}

\linespread{1.4}\zihao{-4}
\bigskip

Nonlinear systems are the main mathematical models for describing natural phenomena. The research of these systems plays an important role in many fields. With the development of high performance computers and computer algebra systems, symbolic computation has become a powerful tool for solving problems related to nonlinear systems. Based on the symbolic computing platform Maple, this paper studies the related automatic algorithms for constructing exact solutions of nonlinear systems. Our work consists of the following three parts.

The first part focuses on the solving of exact solutions to nonlinear systems.

The Hirota bilinear method is an effective method for solving nonlinear differential equations. Based on the Hirota bilinear method, we can construct various types of solutions. However, the $n$-soliton solution formula established from the Hirota bilinear method  only works for integrable equations, and often does not work for non-integrable equations. This paper presented a type of parameter constraints to extend the n-soliton solution formula for non-integrable equations. Based on the Painlevé expansion method, simplified Hirota method, conjugate parameter method as well as  long limit method, an automatic algorithm for constructing soliton, breather and lump solutions of nonlinear evolution equations is proposed, and the corresponding software TwSolver is developed. The TwSolver is applicable to both integrable and non-integrable equations with friendly interfaces. 

Direct algebraic methods are also widely applied when solving differential equations. Its main difficulty is the solving of large-scale nonlinear algebraic equations. In order to address this problem, we designed a grouped and parallel solving algorithm and developed a corresponding software PGSolve. As an application of PGSolve, we further developed the NS1L,  which can automatically construct $n$-soliton and 1-lump interaction solutions. Experiments show that PGSolve can solve equations with a size less than 350 in a few minutes. For equations with a size above 100, its efficiency is more than 100 times higher than that of ordinary solving functions.

The second part mainly studies the $n$-order expansion method and its application.

In the solving of nonlinear differential, many methods are proposed based on the  homogeneous balance principle, such as Painlevé expansion method, tanh function method  and Jacobi elliptic function method. 
These methods usually assume that the required solution is a polynomial of a given function, then determine the order of the required solution by balancing the order of two different highest order terms. 
However, the order of the solution may not be determined when the highest terms have the same order, which is the main reason of loss solutions. 
We found that the order of the required solution will not only appear in the order of the highest $n$ terms, but also in the coefficients of the highest order $n$ terms. 
Therefore, we proposed the $n$-order expansion method, which would balance both orders and coefficients of the highest $n$ terms. The corresponding software is named as NEM. 
Based on NEM, a new algorithm for constructing polynomial solutions of  nonlinear difference equations is proposed, and the corresponding software NLREPS is developed. 
At the same time, this paper also applied the $n$-order expansion method to tanh function method  and Painlevé expansion method. 
Experiments and examples show that the $n$-order expansion method indeed improved the homogeneous balance principle, and can derive more solutions.

The third part mainly studies on the simplification of nonlinear integral expression with respect to abstract functions.

Because the simplification of integral expression is often needed in the process of solving nonlinear differential equations, this paper studies nonlinear integral simplification. Firstly, we established an algebraic system which  regard the integral polynomial as a linear combination of standard integral terms. Then, based on the derivative multiplication rule, a recursive algorithm is designed to find all binomial merge rules. Finally, based on these rules, the simplification problem is transformed into an exact linear programming problem. The corresponding software  IntSimplify is developed to automatically simplify the nonlinear integral expression. IntSimplify can simplify the integral polynomial with nested integrals and redundant terms, and also can simplify more complex integral expressions compared with existing algorithms.

It should be noted that the main line of research in this paper is algorithms for constructing exact solutions of nonlinear differential equations. Constructing polynomial solutions of nonlinear difference equations is the application of the $n$-order expansion method. Integral simplification is an auxiliary tool for the solving of differential equations.

\bigskip
\noindent\textbf{\zihao{4} Keywords:}
exact solution of nonlinear systems, Hirota bilinear method, grouped and parallel solving algorithm, $n$-order expansion method, simplification of nonlinear integral expression
