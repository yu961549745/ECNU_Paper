\documentclass{article}
\usepackage{ctex}
\usepackage{paralist}% 方便好用的压缩列表
\usepackage[a4paper,top=2cm,bottom=2cm,left=2cm,right=2cm,includehead,includefoot]{geometry}
\usepackage{amsmath,amssymb,amsfonts,bm}
\newcommand{\abs}[1]{\left\vert#1\right\vert}
\newcommand{\floor}[1]{\left\lfloor{#1}\right\rfloor}
\newcommand{\ceil}[1]{\left\lceil{#1}\right\rceil}
\newcommand{\sbrace}[1]{\left(#1\right)}
\newcommand{\mbrace}[1]{\left[#1\right]}
\newcommand{\bbrace}[1]{\left\{#1\right\}}
\newcommand{\dbrace}[1]{
  \Bigl\{
    #1
  \Bigr\} 
}
\newcommand{\eval}[2]{\left.{#1}\right|_{#2}}
\newcommand{\conj}[1]{{\rm conj}\sbrace{#1}}
\newcommand{\ALLP}{\mathcal{A}}
\newcommand{\PS}{\mathcal{P}}
\newcommand{\dd}[1]{\mathrm{d}#1}
\newcommand{\ii}[1]{\int\!{#1\dd x}}
\newcommand{\VecNorm}[1]{\left\Vert#1\right\Vert}% 向量模
\newcommand{\spell}[1]{#1}
\newcommand{\up}[1]{^{(#1)}}
\newcommand{\TT}{^\top}% 矩阵转置
\newcommand{\OO}{\ensuremath{\mathbb O}}% n 阶展开多项式余项
\newcommand{\OC}{\ensuremath{\mathcal O}}% 算法复杂度
\newcommand{\lfrac}[2]{#1/#2}
\newcommand{\DIF}[1]{\ensuremath{\frac{\partial}{\partial #1}}}
\newcommand{\DIFF}[2]{\ensuremath{\frac{\partial #1}{\partial #2}}}
\begin{document}
设未知函数为$u=u(x_1,\cdots,x_d)$, 则 $n$-孤子和$m$-lump 解的假设形式为
\begin{equation}
    f_{n,m}=c+\sum_{i=1}^{2^n-1}\mbrace{q_i\prod_{k\in Q_i}{\exp(\xi_k)}}+\sum_{j=1}^{4^m/2-1}\mbrace{s_j\prod_{k \in S_j}{\theta_k}}
\end{equation}
其中 $c$是待定常数, $Q_i$ 是 $\bbrace{1,2,\cdots,n}$ 的第 $i$ 个非空子集, $S_j$ 是 $\bbrace{1,2,\cdots,2m}$ 的第 $j$ 个含偶数个元素的非空子集, $Q_i$和$S_j$都是从本文算法所产生的全体子集中按顺序筛选的. 

\begin{equation}
    \xi_k=p_{1,k}x_1+\cdots+p_{d,k}x_d=\sum_{l=1}^d{p_{l,k}x_l}
\end{equation}
\begin{equation}
    \theta_k=\left\{\begin{array}{ll}
        \sum_{l=1}^d{\sbrace{r_{l,k}+r_{l,k+1}\cdot I}x_l} & k \equiv 1 ({\rm mod~} 2)   \\ 
        \sum_{l=1}^d{\sbrace{r_{l,k-1}-r_{l,k}\cdot I}x_l} & k \equiv 0 ({\rm mod~} 2) \\ 
    \end{array}\right.
\end{equation}
参数的集合为
\begin{equation}
\begin{split}
V&=\bbrace{c}\cup\bbrace{p_{l,k}|1\le l\le d, 1\le k \le n}\cup \bbrace{q_i|1\le i \le 2^n-1}\\
 &\cup\bbrace{r_{l,k}|1\le l \le d,1\le k \le 2m}\cup\bbrace{s_j|1\le j \le 2^{2m-1}-1}
\end{split}
\end{equation}

约束条件为:
\begin{compactitem}[\textbullet]
\item 孤子各项系数非零.
\item LUMP各项系数非零.
\item 整个解的维度没有退化.
\end{compactitem}

从而避免取值集合为
\begin{equation}
\begin{split}
F&=\bbrace{\bbrace{q_i=0}|1\le i \le 2^n-1} \\
&\cup\bbrace{\bbrace{s_j=0}|1\le j \le 2^{2m-1}-1} \\
&\cup\bbrace{\bbrace{p_{l,1}=0,\cdots,p_{l,n}=0,r_{l,1}=0,\cdots,r_{l,2m}=0}|1\le l \le d}
\end{split}
\end{equation}

形式很美好, 但是 lump 部分次数太高了, 即使 1-lump , YTSF 不令 $s_1=1$ 也解不了. 

\end{document}