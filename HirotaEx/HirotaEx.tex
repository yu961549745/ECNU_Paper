\documentclass[12pt,a4paper,UTF8]{article}
\usepackage{ctex}% 中文支持
\usepackage{paralist}% 方便好用的压缩列表
\usepackage[a4paper,top=2cm,bottom=2cm,left=2cm,right=2cm,includehead,includefoot]{geometry}
\usepackage{amsmath,amssymb,amsfonts,bm}
\usepackage[colorlinks]{hyperref}
\usepackage{indentfirst}
\usepackage{textcomp}
\usepackage{pbox}
\numberwithin{equation}{section}

%----------------  自定义命令 ---------------------------------------
\newcommand{\abs}[1]{\left\vert#1\right\vert}
\newcommand{\floor}[1]{\left\lfloor{#1}\right\rfloor}
\newcommand{\ceil}[1]{\left\lceil{#1}\right\rceil}
\newcommand{\sbrace}[1]{\left(#1\right)}
\newcommand{\mbrace}[1]{\left[#1\right]}
\newcommand{\bbrace}[1]{\left\{#1\right\}}
\newcommand{\dbrace}[1]{
  \Bigl\{
    #1
  \Bigr\} 
}
\newcommand{\eval}[2]{\left.{#1}\right|_{#2}}
\newcommand{\conj}[1]{{\rm conj}\sbrace{#1}}
\newcommand{\ALLP}{\mathcal{A}}
\newcommand{\PS}{\mathcal{P}}
\newcommand{\dd}[1]{\mathrm{d}#1}
\newcommand{\ii}[1]{\int\!{#1\dd x}}
\newcommand{\VecNorm}[1]{\left\Vert#1\right\Vert}% 向量模
\newcommand{\spell}[1]{#1}
\newcommand{\up}[1]{^{(#1)}}
\newcommand{\TT}{^\top}% 矩阵转置
\newcommand{\OO}{\ensuremath{\mathbb O}}% n 阶展开多项式余项
\newcommand{\OC}{\ensuremath{\mathcal O}}% 算法复杂度
\newcommand{\lfrac}[2]{#1/#2}
\newcommand{\DIF}[1]{\ensuremath{\frac{\partial}{\partial #1}}}
\newcommand{\DIFF}[2]{\ensuremath{\frac{\partial #1}{\partial #2}}}
\renewcommand{\mod}{{\rm ~mod~}}
\newcommand{\FM}{\mathfrak{M}}
\newcommand{\FN}{\mathfrak{N}}
\newcommand{\CF}{\mathcal{F}}
\newcommand{\SP}{\mathfrak{S}} % 下标替换函数

\newcommand{\Painleve}{Painlev{\'e}}
\newcommand{\refeqn}[1]{Eq. (\ref{#1})}
\newcommand{\refsec}[1]{Section \ref{#1}~}
\newcommand{\reftab}[1]{Table \ref{#1}~}
\newcommand{\MLNS}{$m$-lump和$n$-孤子的相互作用解}
%--------------------------------------------------------------------

\begin{document}
\title{Constructing high order interaction solutions \\ base on the extended Hirota method}
\author{余江涛}
\maketitle
\begin{abstract}
在 Hirota 方法的 $n$-孤子解公式的基础上, 基于长极限方法的思想, 令解中的部分行波趋于零, 得到了$m$-lump和$n$-孤子相互作用解的公式. 同时, 引入了参数约束条件, 使得该公式对不可积方程也成立, 提出了拓展的 Hirota 方法, 可用于求任意阶的孤子, 呼吸子和lump的相互作用解. 以XXX方程和YYY方程为例, 我们基于拓展的 Hirota 方法, 计算了这些方程的高阶相互作用解.

\vspace{0.2cm}

{\noindent\bf Keywords:} Hirota 方法, 长极限方法, 相互作用解
\end{abstract}

\section{Introduction}
\begin{compactitem}[\textbullet]
\item 介绍 Hirota 方法及其后续工作, 特别是长极限方法. 
\item 介绍相互作用解的相关工作, 估计只有直接代数方法. 指出主要困难是大规模代数方程组的求解, 主要问题是无法求解高阶的解.
\item 我们在 Hirota 方法的 $n$-孤子解公式的基础上, 基于长极限方法的思想, 令解中的部分行波趋于零, 得到了$m$-lump和$n$-孤子相互作用解的公式. 基于共轭赋值的方法, 又能将部分孤子转化为呼吸子, 所以该公式可用于求任意阶的孤子, 呼吸子和lump的相互作用解.
\item 因为 Hirota 方法的$n$-孤子解公式只对可积方程成立, 对不可积方程往往不成立, 所以我们的公式也存在同样的问题. 为了解决这个问题, 我们引入了一种参数约束, 使得我们的方法对不可积方程也适用. 
\end{compactitem}

本文的组织结构如下: 首先在\refsec{method}中介绍拓展Hirota方法的整体流程; 然后在\refsec{proof}中详细推导$m$-lump与$n$-孤子相互作用解的公式; 接着在\refsec{example}中以XXX方程和YYY方程为例求高阶相互作用解; 在\refsec{experiment}中探索参数的约束条件对解的影响; 最后在\refsec{conclusion}中进行总结. 

\section{Extended Hirota Method}\label{method}

对于一个$(d+1)$维的未知函数$u=u(x_1,\cdots,x_d,t)$, 关于它的非线性演化方程(简称 NLEE)是一个关于$u$及其导数的多项式方程, 即
\begin{equation}
    U(u,u\up{1},u\up{2},u\up{3}\cdots)=0, \label{oeq}
\end{equation}
其中$u\up{k}~(k=1,2,3,\cdots)$表示所有的$k$阶导数. 例如, $u\up{1}=\bbrace{u_t,u_{x_1},\cdots,u_{x_n}}$.

基于\Painleve{}展开法, 可以得到其\Painleve{}截断展开(简称 TPE)
\begin{equation}
u=\sum_{k=1}^{\alpha}{\frac{u_k}{f^{\alpha-k+1}}}. 
\end{equation}
将其代入原方程可以得到一个关于$f$及其导数的多项式方程 
\begin{equation}
F\sbrace{f,f\up{1},f\up{2},\cdots}=0.
\end{equation}

对于$u=u\sbrace{x_1,\cdots,x_d,t}$, 其行波有多种假设形式. 例如(此处可引用许多文献), 
\begin{subequations}
\begin{align}
\xi&=p_1 x_1 + p_2 x_2 + \cdots + p_d x_d + \omega t+p_{d+1},  \label{xi-1}\\ 
\xi&=p_1 (x_1 + p_2 x_2 + \cdots + p_d x_d + \omega t)+p_{d+1}, \label{xi-2}\\ 
\xi&=p_0 \sbrace{p_1 x_1 + p_2 x_2 + \cdots + p_d x_d + \omega t}+p_{d+1}, \label{xi-3}\\
\xi&=p_0 \sbrace{p_1 x_1 + p_2 x_2 + \cdots + p_d x_d + \omega t+p_{d+1}}. \label{xi-4} 
\end{align}
\end{subequations}
其中
\begin{compactitem}[\textbullet]
\item \refeqn{xi-1}只能用于求孤子解, 不能用于求 lump 解.
\item \refeqn{xi-2}和\refeqn{xi-3}可以用于求孤子解和lump解. 
\item 用\refeqn{xi-3}求解, 其形式更加复杂. 对于不可积方程, 其解成立的条件也更复杂.
\item \refeqn{xi-3}能够涵盖\refeqn{xi-1}和\refeqn{xi-2}, 是最一般的假设形式.
\item \refeqn{xi-4}和\refeqn{xi-3}基本是等价的, 但是\refeqn{xi-4}能够使得lump解的各个行波也有常数项.
\end{compactitem}
所以我们采用\refeqn{xi-4}作为行波的假设形式. 最终, 我们取行波
\begin{equation}
  \xi=p_0 \sbrace{p_1 x_1 + p_2 x_2 + \cdots + p_d x_d + \omega t+p_{d+1}}, 
  \label{xi}
\end{equation}
同时, 我们在下文中简记$p_0=\delta$. 

为了解决Hirota方法中$n$-孤子解公式对不可积方程往往不成立的问题, 我们提出了参数下标集合$\PS$: 
\begin{equation}
\PS\subseteq \ALLP=\bbrace{1,2,\cdots,d,d+1} ,
\end{equation}
同时, 我们定义下标替换函数
\begin{equation}
\SP\sbrace{e,i;\PS}=\left\{\begin{array}{ll}
  p_k \to p_{k,i}, & k \in \PS \cup \bbrace{0},\\ 
  p_k \to p_k , & k \not\in\PS \cup \bbrace{0}.
\end{array}\right.
\end{equation}
需要注意的是, 始终有$p_0\to p_{0,i}$, 即$\delta\to \delta_i$. 

$\PS$在我们的算法中是非常关键的. $\PS= \ALLP$表示算法会根据孤子解的公式生成没有参数约束的解. 本文引入$\PS$的理由如下: 
\begin{compactenum}[1. ]
\item 对于一些不可积的方程, 取$\PS= \ALLP$时得到的解不能满足原方程. 但是, 取$\PS\subsetneq  \ALLP$会得到满足原方程的解.
\item 尽管$\PS\subsetneq  \ALLP$可以看作是$\PS= \ALLP$的特例, 但是有一些方程会在$\PS= \ALLP$时无法求解$h_{i,j}$.
\item 对于一些不可积的方程, 我们发现若约束某些$p_{i,k}~(k=1,2,\cdots,m)$ 取值相同就能够得到满足原方程的解, 而$\PS$ 就是为了指定这类约束.
\end{compactenum}

此外, $d+1$对应的是行波中的常数项, 它的存在不会影响解的正确性, 但是$d+1\in \PS$能够使得每个行波都有一个独立的平移量.

在 Hirota 方法中, $n$-孤子解的公式为 
\begin{equation}
f_n=\sum_{\mu=0,1}\exp\sbrace{\sum_{i=1}^m{\mu_i \xi_i}+\sum_{1\le i<j\le m}{\mu_i\mu_jH_{i,j}}}.
\label{soliton-old}
\end{equation}
其中, 求和范围$\mu=0,1$表示对所有可能的$\mu_k\in \bbrace{0,1}$进行求和. 所以, 它的求和范围是$\sbrace{\mu_1,\cdots,\mu_m}\in \bbrace{0,1}^m$.

为了便于理解, 同时便于推导\MLNS{}的公式, 我们将其改写为
\begin{equation}
f_n=\sum_{T\subseteq \bbrace{1,\cdots,n}}\mbrace{\sbrace{\prod_{\bbrace{i,j}\subseteq T}{h_{i,j}}}\exp\sbrace{\sum_{k\in T}{\xi_k}}}, 
\label{soliton-new}
\end{equation}
其中$h_{i,j}=\exp(H_{i,j})$. 需要说明的是, 当求和范围为空时, 结果为0; 当连乘范围为空时, 结果为1. 例如, 我们有
\begin{equation}
\begin{aligned}
f_1&=1+\exp(\xi_1) , \\ 
f_2&=1+\exp(\xi_1)+\exp(\xi_2)+h_{12}\exp(\xi_1+\xi_2) ,\\ 
f_3&=1+\exp(\xi_1)+\exp(\xi_2)+h_{12}\exp(\xi_1+\xi_2) ,\\ 
   &+\exp(\xi_3)+h_{13}\exp(\xi_1+\xi_3)+h_{23}\exp(\xi_2+\xi_3) \\
   &+h_{12}h_{13}h_{23}\exp(\xi_1+\xi_2+\xi_3) .
\end{aligned}
\end{equation}
在不引起歧义的情况下,我们会将类似于$h_{i,j}$的双下标简写为类似于$h_{12}$的形式. 

从$1+\exp(\xi)=0$中解得色散关系$\omega=\omega(p_0,\cdots,p_{d+1})$后, 代入\refeqn{xi}中的行波, 可得
\begin{equation}
  \xi_i=\SP\sbrace{\xi,i;\PS} , ~ \xi_j=\SP\sbrace{\xi,j;\PS} .
\end{equation}
然后, 将$\xi_i,\xi_j$代入$1+\exp(\xi_i)+\exp(\xi_j)+h_{i,j}\exp(\xi_i+\xi_j)=0$, 可以解得相互作用系数$h_{i,j}$. 若$\omega$和$h_{i,j}$都有解, 则能够基于\refeqn{soliton-new}得到原方程的$n$-孤子解; 否则, 算法结束. 

上述就是基于Hirota方法求$n$-孤子解的过程. 相比于传统的Hirota方法, 我们的方法唯一的区别在于利用参数下标集合$\PS$给出了参数的约束条件. 若一个方程能够基于Hirota方法求得$n$-孤子解, 则它就能基于我们的方法进一步求得\MLNS{}. 

设$I$表示虚数单位, 在$2m+n$孤子的公式中, 令
\begin{equation}
\xi_k=\left\{\begin{array}{ll}
  \SP\sbrace{\delta\sbrace{p_1 x_1+\cdots+p_d x_d+\omega t+p_{d+1}} +\pi I,k;\PS} ,&  1\le k\le 2m,\\
  \SP\sbrace{\delta\sbrace{p_1 x_1+\cdots+p_d x_d+\omega t+p_{d+1}},k;\PS} ,&  2m+1\le k \le 2m+n, \\
\end{array}\right. 
\end{equation}
当$\delta_k\to 0~(k=1,2,\cdots,2m)$, 可以得到展开
\begin{equation}
\begin{aligned}
  \exp\sbrace{\xi_k}&=\left\{\begin{array}{ll}
    -\sbrace{1+\delta_k \theta_k}+o\sbrace{\delta_k}, & k\le 2m, \\ 
    \exp\sbrace{\xi_k}, & k>2m,   
    \end{array}\right. \\ 
  h_{i,j}&=\left\{\begin{array}{ll}
    1+\delta_i \delta_j b_{i,j}+o\sbrace{\delta_i \delta_j}, & i<j\le 2m, \\
    1+\delta_i \psi_{i,j}+o\sbrace{\delta_i}, & i\le 2m < j , \\
    h_{i,j}, & 2m<i<j.
    \end{array}\right.
\end{aligned}
\label{expand}
\end{equation}
事实上, 我们有 
\begin{equation}
\begin{aligned} 
\theta_i &= \eval{\DIFF{\xi_i}{\delta_i}}{\delta_i=0},\\ 
\psi_{i,j} &= \eval{\DIFF{h_{i,j}}{\delta_i}}{\delta_i=0} ,\\ 
b_{i,j} &= \eval{\frac{\partial^2 h_{i,j}}{\partial \delta_i \partial \delta_j}}{\delta_i=0,\delta_j=0} .
\end{aligned}
\label{params}
\end{equation}
我们将在\refsec{proof}中证明这个结论. 

将\refeqn{expand}代入$f_{2m+n}$就能得到\MLNS{}的公式. 令
\begin{equation}
\begin{aligned}
M&=\bbrace{1,\cdots,2m}, \\ 
N&=\bbrace{2m+1,\cdots,2m+n}, \\ 
\end{aligned}
\end{equation}
设
\begin{equation}
\mathcal{L}(M)=\bbrace{\mbrace{s_1,\cdots,s_{2l}}|s_{2k}>s_{2k-1},s_{2k+1}>s_{2k-1},s_k\in M,l=0,\cdots,m} , 
\end{equation}
则\MLNS{}的公式为
\begin{equation}
f_{m,n}=\sum_{T\subseteq N}\mbrace{H(T)E(T)\sum_{S\in \mathcal L(M)}{B(S)\Psi(M-S,\bbrace{0}\cup T)}}.
\label{fmn}
\end{equation}
其中 
\begin{equation}
\begin{aligned}
H(T)&=\prod_{\bbrace{i,j}\subseteq T}{h_{i,j}}, \\ 
E(T)&=\exp\sbrace{\sum_{k\in T}{\xi_k}}=\prod_{k\in T}\exp(\xi_k), \\ 
B(S)&=\prod_{k=1}^{|S|/2}{b_{s_{2k-1},s_{2k}}},\\ 
\Psi(S,T)&=\sum_{j_k\in T}{\prod_{k=1}^{|S|}{\psi_{s_k,j_k}}}, ~(\psi_{i,0}=\theta_i). 
\end{aligned}
\end{equation}
其中, $|\cdot|$用于表示集合或序列中的元素个数. 需要说明的是, $S$是有序的, 只有在计算$M-S$时, 它才被当做集合. 

我们将在\refsec{proof}中证明\refeqn{fmn}的正确性. 现在我们以$m=1,n=1$为例, 来展示\refeqn{fmn}的计算方法. 当$m=1,n=1$时, 
\begin{equation}
M=\bbrace{1,2},N=\bbrace{3},
T\in\bbrace{\emptyset,\bbrace{3}}, 
S\in\bbrace{[~],[1,2]},
\end{equation}
我们有 
\begin{equation}
\begin{aligned}
f_{1,1}&=\mbrace{B([~])\Psi(\bbrace{1,2},\bbrace{0})+B([1,2])\Psi(\emptyset,\bbrace{0})}H(\emptyset)E(\emptyset) \\ 
&+\mbrace{B([~])\Psi(\bbrace{1,2},\bbrace{0,3})+B([1,2])\Psi(\emptyset,\bbrace{0,3})}H(\bbrace{3})E(\bbrace{3}).
\end{aligned}
\end{equation}
根据连乘和求和的约定, 我们有
\begin{equation}
  B([~])=\Psi(\emptyset,T)=H(\emptyset)=E(\emptyset)=H(\bbrace{3})=1. 
\end{equation}
此外, 
\begin{equation}
\begin{aligned}
E(\bbrace{3})&=\exp(\xi_3), \\ 
B([1,2])&=b_{12}, \\
\Psi(\bbrace{1,2},\bbrace{3})&=\psi_{10}\psi_{20}=\theta_1 \theta_2,\\ 
\Psi(\bbrace{1,2},\bbrace{0,3})&=\psi_{10}\psi_{20}+\psi_{13}\psi_{20}+\psi_{10}\psi_{23}+\psi_{13}\psi_{23} \\
&=\theta_1\theta_2+\psi_{13}\theta_2+\theta_1\psi_{23}+\psi_{13}\psi_{23}.\\ 
\end{aligned}
\end{equation}
所以, 我们有
\begin{equation}
\begin{aligned}
f_{1,1}&=\theta_1 \theta_2+b_{12}+\exp(\xi_3)\sbrace{\theta_1\theta_2+\psi_{13}\theta_2+\theta_1\psi_{23}+\psi_{13}\psi_{23}+b_{12}}. 
\end{aligned}
\label{f11-new}
\end{equation}
可以看出, 本文给出的相互作用解和以往的相互作用解有着本质上的差别. 以往在计算相互作用解的过程中, 我们一般假设
\begin{equation}
\begin{aligned}
f_{1,1}&=\theta_1 \theta_2+c+h \exp(\xi_3) \\ 
&=\xi_1^2+\xi_2^2+c+h \exp(\xi_3) . 
\end{aligned}
\label{f11-old}
\end{equation}
因为$\theta_1=\theta_2^*$ ($^*$表示共轭), 所以$\theta_1 \theta_2$能够写成两个行波的平方和. 在\refeqn{f11-old}中, 并没有指数和行波的乘法项, 而在我们的方法中, 指数和行波的乘法项也会出现在lump和孤子的相互作用解中. 

若展开$f_{m,n}$中所有的括号, 则其中加法项的个数应为
\begin{equation}
\begin{aligned}
\#(m,n)&=\sum_{i=0}^n\mbrace{\binom{n}{i}\sum_{j=0}^m{\binom{2m}{2j}\frac{(2j)!}{j!2^j}(i+1)^{2m-2j}}}. 
\end{aligned}
\label{n-fmn}
\end{equation}
列出上述公式的理由为:
\begin{compactitem}[\textbullet]
\item 设$|T|=i$, 则对应$\binom{n}{i}$个不同的集合.
\item $S$的长度为$2j$, 则共有 $\binom{2m}{2j}\frac{(2j)!}{j!2^j}$ 可能的$S$. 其中$\binom{2m}{2j}(2j)!$表示从$2m$个数中选择$2j$个进行全排列; 除以$2^n$表示只保留$i<j$的$b_{i,j}$, 能够满足$s_{2k}>s_{2k-1}$; 除以$j!$表示在$b_1,\cdots,b_j$的全排列中只保留$s_{2k+1}>s_{2k-1}$的排列. 
\item 因为$\psi$的第二个下标有$|T\cup \bbrace{0}|^{|M-S|}$种可能, 所以当$|T|=i,|S|=j$时, $\Psi(M-S,T\cup\bbrace{0})$中有$(i+1)^{2m-2j}$项.
\end{compactitem}
综上, 我们得出了\refeqn{n-fmn}. 当$m\le 4, n\le 6$时, $\#(m,n)$的结果如\reftab{tb-n-fmn}所示. 可以看出$\#(m,n)$的增长非常快.

\begin{table}[htbp]
\centering 
\caption{\MLNS{}的生成公式中的项数} \label{tb-n-fmn}
\begin{tabular}{c|*{7}{r}}
\hline 
$m\backslash n$ & 0 & 1 & 2 & 3 & 4 & 5 & 6\\
\hline 
0 & 1 & 2 & 4 & 8 & 16 & 32 & 64 \\
1 & 2 & 7 & 22 & 64 & 176 & 464 & 1184 \\
2 & 10 & 53 & 234 & 908 & 3208 & 10560 & 32896 \\
3 & 76 & 575 & 3438 & 17336 & 77080 & 311352 & 1166320 \\
4 & 764 & 7957 & 63018 & 406756 & 2249960 & 11046768 & 49351904 \\
\hline 
\end{tabular}
\end{table}

在基于\refeqn{fmn}得到\MLNS{}的公式之后, 为了获得正确的lump解, 还需满足
\begin{equation}
  \theta_{2k-1}=\theta_{2k}^*, ~(k=1,\cdots,m).
\end{equation}
此外, 令$n=2\kappa+\tau$, 取
\begin{equation}
  \xi_{2m+2k-1}=\xi_{2m+2k}^*, ~(k=1,\cdots,\kappa),
\end{equation}
还能使得前$2\kappa$个孤子变为呼吸子. 所以, 取
\begin{equation}
p_{k,i}=\left\{
\begin{array}{ll}
  p_{k,i,RE}+I\cdot p_{k,i,IM}, & i\le 2(m+\kappa), i\equiv(1 \mod 2),\\
  p_{k,i-1,RE}-I\cdot p_{k,i-1,IM}, & i\le 2(m+\kappa), i\equiv(0 \mod 2),\\
  p_{k,i}, & 2(m+\kappa)+1 \le i \le 2(m+\kappa)+\tau ,
\end{array}
\right.
\end{equation}
可以得到$m$-lump, $\kappa$-呼吸子 和 $\tau$-孤子的相互作用解. 这里$m,\kappa,\tau\ge 0$, 所以取其中两个为零, 另外一个不为零也能得到单独的解. 最终, 解中所有的参数构成集合
\begin{equation}
\begin{aligned}
\mathbb{P}
&=\bbrace{p_k|k\not\in \PS\cup\bbrace{0}} \\
&\cup\bbrace{p_{k,2i-1,RE},p_{k,2i-1,IM}|k\in\PS\cup\bbrace{0},1\le i \le m+\kappa} \\
&\cup\bbrace{p_{k,i}|k\in\PS\cup\bbrace{0},2(m+\kappa)+1 \le i \le 2(m+\kappa)+\tau},
\end{aligned}
\label{xi-params}
\end{equation}
且$\mathbb P$中所有的参数取值均为实数. 

综上所述, 拓展的 Hirota 方法的基本步骤可以总结为: 
\begin{compactenum}[Step 1.]
\item 基于\Painleve{}展开法计算输入的NLEE的 TPE, 将NLEE转化为关于$f$及其导数的多项式方程. 
\item 给定$\PS$, 基于1-孤子和2-孤子的假设形式, 求得色散关系$\omega$和相互作用系数$h_{i,j}$. 若无法求解则修改$\PS$重新求解. 
\item 基于\refeqn{params}计算其它关键参数. 若在计算是出现了分母为零等异常情况, 则表明该方程在当前$\PS$下没有lump解和与lump解相互作用的解, 只存在孤子和呼吸子相互作用的解. 若仍要构造lump解, 则需修改$\PS$重新求解. 
\item 基于\refeqn{fmn}生成\MLNS{}的公式, 并将相关参数代入. 
\item 取$n=2\kappa+\tau$, 根据\refeqn{xi-params}对行波中的参数进行赋值.
\item 将上一步的结果代入TPE, 就能得到$m$-lump, $\kappa$-呼吸子 和 $\tau$-孤子的相互作用解.
\item 代入验证所得的解, 若满足原方程则算法结束; 否则, 就修改$\PS$重新求解. 
\end{compactenum}

一般而言, 若所得的3-孤子解能够满足原方程, 则同一$\PS$下的其它高阶解也能满足原方程. 

\section{Proof}\label{proof}
在本节中, 我们将证明\refeqn{params}和\refeqn{fmn}的正确性.

首先, 我们来证明\refeqn{params}的正确性. 当$\delta\to 0$时, 令$\eta(\delta)=\delta(p_1 x_1 +\cdots + p_d x_d + \omega(\delta)t+p_{d+1})$, 则有一维泰勒展开
\begin{equation}
\begin{aligned}
\exp(\eta(\delta)+\pi I)&=\exp(\pi I)\exp(\eta(\delta)) \\ 
&=-\sbrace{\exp(\eta(0))+\eta'(0)\exp(\eta(0))\delta+o(\delta)} \\ 
&=-(1+\eta'(0))\delta+o(\delta) , 
\end{aligned}
\end{equation}
故
\begin{equation}
  \theta_i = \eval{\DIFF{\xi_i}{\delta_i}}{\delta_i=0}.
\end{equation}

令$h_{i,j}=h(\delta_i,\delta_j)$, 根据对称性我们有$h(x,y)=h(y,x)$. 此外, 取$\delta_2=0$可以将2-孤子解退化为1-孤子解, 所以$h(\delta_1,0)=1$. 根据对称性, 我们有$h(0,\delta_2)=1$. 从而, $h(0,0)=1$, 且
\begin{equation}
    h_x(0,0)=\eval{\frac{\partial}{\partial x}h(x,0)}{x=0}=0.
\end{equation}
类似地, 我们有$h_y(0,0)=h_{xx}(0,0)=h_{yy}(0,0)=0$. 因此, $h(x,y)$在$(0,0)$点的二维泰勒展开为 
\begin{equation}
\begin{aligned}
h(x,y)&=h(0,0)+h_x(0,0)x+h_y(0,0)y \\ 
&+\frac{1}{2}\mbrace{h_{xx}(0,0)x^2+2h_{xy}(0,0)xy+h_{yy}(0,0)y^2}+o(x^2+y^2) \\ 
&=1+h_{xy}(0,0)xy+o(xy),
\end{aligned}
\end{equation}
故
\begin{equation}
  b_{i,j} = \eval{\frac{\partial^2 h_{i,j}}{\partial \delta_i \partial \delta_j}}{\delta_i=0,\delta_j=0}. 
\end{equation}
当仅有$x\to 0$时, 
\begin{equation}
\begin{aligned}
  h(x,y)&=h(0,y)+h_x(0,y)x+o(x) \\ 
  &=1+h_x(0,y)x+o(x),
\end{aligned}
\end{equation}
故
\begin{equation}
  \psi_{i,j} = \eval{\DIFF{h_{i,j}}{\delta_i}}{\delta_i=0}. 
\end{equation}
综上所述, \refeqn{params}是正确的. 

然后, 我们来证明\refeqn{fmn}的正确性. 为了方便证明, 我们将$n$-孤子解的公式拓展为
\begin{equation}
  \CF(N)=\sum_{T\subseteq N}{H(T)E(T)}.
\end{equation}
即将其视为关于集合$N$的函数, 当$N=\bbrace{1,\cdots,n}$时就是$n$-孤子解. 

当$M=\bbrace{1,\cdots,2m},N=\bbrace{2m+1,\cdots,2m+n}$时, 记
\begin{equation}
\begin{aligned}
  &\FM=M\cup N = \bbrace{1,2,\cdots,2m+n}, \\ 
  &\FN(i)=\FM - \bbrace{i},
\end{aligned}
\end{equation}
则有
\begin{equation}
\begin{aligned}
  \CF(\FM)&=\sum_{T\subseteq \FM}{H(T)E(T)}=\sum_{T\subseteq \FN(i)\cup \bbrace{i}}{H(T)E(T)} \\
  &=\sum_{T\subseteq \FN(i)}{H(T)E(T)}+\sum_{T\subseteq \FN(i)}{H(T\cup \bbrace{i})E(T\cup \bbrace{i})} \\ 
  &=\CF(\FN(i))+\exp(\xi_i)\sum_{T\subseteq \FN(i)}\mbrace{H(T)E(T)\prod_{k\in T}{h_{k,i}}}. 
\end{aligned}
\end{equation}
一个简单的例子是
\begin{equation}
\begin{aligned}
\CF(\bbrace{1,2,3})&=1+\exp(\xi_1)+\exp(\xi_2)+h_{12}\exp(\xi_1+\xi_2)  \\ 
&+\exp(\xi_3)\sbrace{1+h_{13}\exp(\xi_1)+h_{23}\exp(\xi_2)+h_{13}h_{23}\exp(\xi_1+\xi_2)}.
\end{aligned}
\end{equation}

若$i\in M,T\subseteq \FN(i)$, 则有
\begin{equation}
\begin{aligned}
  \exp(\xi_i)&=-(1+\delta_i \theta_i)+o(\delta_i), \\ 
  \prod_{k\in T}{h_{k,i}}&=\prod_{k\in T}\sbrace{1+\delta_i \psi_{i,k}+o(\delta_i)} ,
\end{aligned}
\end{equation}
故
\begin{equation}
  \eval{\CF(\FM)}{\delta_i=0}=\CF(\FN(i))-\sum_{T\subseteq \FN(i)}{H(T)E(T)}=0. 
\end{equation}

因为上式对所有的$i\in M$都成立, 记$\Delta(M)=\prod_{k\in M}{\delta_k}$, 我们有
\begin{equation}
\begin{aligned}
  \CF(\FM)&=\sbrace{\prod_{k=1}^{2m}{\delta_k}}\sbrace{f_{m,n}+o(1)}\\
  &=\Delta(M)f_{m,n}+o\sbrace{\Delta(M)}. 
\end{aligned}
\label{fmn-o}
\end{equation}
因为$\Delta(M)$是一个能够被TPE消去的常数因子, 所以我们以$f_{m,n}$作为\MLNS{}的生成公式. 现在, 我们只要从$\CF(\FM)$中提取恰好以$\Delta(M)$作为因子的项就能得到$f_{m,n}$.

将$T$按照和$M,N$的关系进行分类, 我们有
\begin{equation}
\begin{aligned}
  H(T)&=\pi_1(T)\pi_2(T)H(T\cap N), \\ 
  E(T)&=\pi_3(T)E(T\cap N),
\end{aligned}
\end{equation}
其中
\begin{equation}
\begin{aligned}
  \pi_1(T)&=\prod_{\bbrace{i,j}\subseteq T\cap M}\sbrace{1+\delta_i \delta_j b_{i,j}+o(\delta_i \delta_j)} ,\\
  \pi_2(T)&=\prod_{i\in T\cap M,j\in T\cap N}\sbrace{1+\delta_i \psi_{i,j}+o(\delta_i)} ,\\
  \pi_3(T)&=\prod_{k\in T\cap M}\sbrace{-1-\delta_k \theta_k+o(\delta_k)} \\
  &=(-1)^{|T\cap M|}\prod_{k\in T\cap M}\sbrace{1+\delta_k \theta_k+o(\delta_k)}.\\
\end{aligned}
\label{de-prod}
\end{equation}
显然, 只要从$\pi_0(T)=\pi_1(T)\pi_2(T)\pi_3(T)$中选择恰好以$\Delta(M)$作为因子的项, 就能得到$f_{m,n}$. 

从\refeqn{de-prod}中不难看出, 要想提取出以$\Delta(M)$作为因子的项, 必须有$T\cap M=M$. 此时, 可以按照如下步骤提取以$\Delta(M)$作为因子的项.

首先, 从$\pi_1(T)$中提取$\delta_i \delta_j b_{i,j}$的乘积, 设乘法中$b_{i,j}$的下标构成序列$S$, 且$i<j$, $b_{i,j}$按照第一个下标升序排列, 则显然$S\in \mathcal L(M)$, 这部分的乘积可以记为$\Delta(S)B(S)$. 

然后, 还需要从$\pi_2(T)\pi_3(T)$中提取含$\Delta(M-S)$的项, 为了统一$\psi$和$\theta$, 我们令$\psi_{k,0}=\theta_k$, 则这部分的乘积可以表示为$\Delta(M-S)\Psi(M-S,\bbrace{0}\cup (T\cap N))$. 此外, 还需要说明的是, 因为$T\cap M=M$, 故提取的结果中不含负号. 

从而, 当$T\cap M=M$时有
\begin{equation}
  \pi_0(T)=\Delta(M)\sum_{S\in \mathcal L(M)}{B(S)\Psi(M-S,\bbrace{0}\cup (T\cap N))}+o(\Delta(M)),
\end{equation}
否则, $\pi_0(T)=0+o(\Delta(M))$.

最终, 简记$T_N=T\cap N$, 我们有
\begin{equation}
\begin{aligned}
\CF(\FM)&=\sum_{T\subseteq \FM}{H(T_N)E(T_N)\pi_0(T)} \\ 
&=\Delta(M)\sum_{M\subseteq T\subseteq M\cup N}\mbrace{H(T_N)E(T_N)\sum_{S\in \mathcal L(M)}{B(S)\Psi(M-S,\bbrace{0}\cup T_N)}}+o(\Delta(M)) \\
&=\Delta(M)\sum_{T\subseteq N}\mbrace{H(T)E(T)\sum_{S\in \mathcal L(M)}{B(S)\Psi(M-S,\bbrace{0}\cup T)}}+o(\Delta(M)) .
\end{aligned}
\end{equation}
故
\begin{equation}
f_{m,n}=\sum_{T\subseteq N}\mbrace{H(T)E(T)\sum_{S\in \mathcal L(M)}{B(S)\Psi(M-S,\bbrace{0}\cup T)}}. 
\end{equation}
证毕. 

\section{Examples}\label{example}

\section{Experiments}\label{experiment}
\begin{table}[htbp]
\newcommand{\mb}[1]{\makebox[1.05em][c]{#1}}
\newcommand{\V}{\mb{\checkmark}}
\newcommand{\X}{\mb{\texttimes}}
\centering
\caption{解的验证结果}
\begin{tabular}{l*{7}{c}}
\hline
\multicolumn{1}{c}{方程名称} & 1 & 2 & 12 & 3 & 13 & 23 & 123 \\ 
\hline
(1+1)KdV & \V\mb{1} &  &  &  &  &  &  \\
(2+1)BKP-T & \V\V & \V\V & \V\V &  &  &  &  \\
(2+1)KP & \V\V & \V\V & \V\V &  &  &  &  \\
(2+1)SK & \V\V & \V\V & \V\V &  &  &  &  \\
(4+1)Fokas-T-2 & \V\V & \V\V & \V\V &  &  &  &  \\
(2+1)CBS & \V\mb{1} & \V\mb{1} & \V\mb{1} &  &  &  &  \\
(2+1)CBS-G & \X\X & \X\X & \X\X &  &  &  &  \\
(3+1)BKP & \V\V & \V\V & \X\X & \V\V & \X\X & \X\X & \X\X \\
(3+1)KP & \V\V & \V\V & \X\X & \V\V & \X\X & \X\X & \X\X \\
(3+1)JM & \V\mb{1} & \V\V & \X\X & \V\mb{1} & \X\X & \X\X & \X\X \\
(3+1)NEE-T & \V\mb{1} & \V\V & \X\X & \V\mb{1} & \X\X & \X\X & \X\X \\
(4+1)Fokas-T & \V\V & \V\mb{1} & \X\X & \V\mb{1} & \X\X & \X\X & \X\X \\
(3+1)YTSF & \mb{2}\mb{2} & \V\V & \mb{2}\mb{2} & \V\mb{1} & \mb{2}\mb{2} & \mb{2}\mb{2} & \mb{2}\mb{2} \\
(3+1)CBS & \V\mb{1} & \V\mb{1} & \V\mb{1} & \V\mb{1} & \V\mb{1} & \V\mb{1} & \V\mb{1} \\
\hline
\multicolumn{8}{c}{\pbox{0.7\textwidth}{
{\tiny ~}\\
注: 123是$\PS=\bbrace{1,2,3}$的简写形式. 每个单元格第一个元素对应3-孤子解的验证结果, 第二个元素对应2-lump解的验证结果. \V 表示经验证是解, \X 表示经验证不是解, 1表示求lump解的相关系数时分母为零, 2表示$h_{i,j}$ 无解. 表中出现的方程如果未出现在本章中, 则其表达式在XXX中.
}}
\end{tabular}
\end{table}

\section{Conclusions}\label{conclusion}

\newpage 
\section*{Appendix}
常用生成公式
\begin{equation*}
\begin{aligned}
f_{0,1}&=1+{\rm e}^{{{\xi}_{1}}} ,\\ 
f_{0,2}&=1+{\rm e}^{{{\xi}_{1}}}+{\rm e}^{{{\xi}_{2}}}+{{h}_{12}}\,{\rm e}^{{{\xi}_{1}}}\,{\rm e}^{{{\xi}_{2}}} ,\\ 
f_{1,0}&={{\theta}_{1}}\,{{\theta}_{2}}+{{b}_{12}} ,\\ 
f_{2,0}&={{\theta}_{1}}\,{{\theta}_{2}}\,{{\theta}_{3}}\,{{\theta}_{4}}+{{b}_{12}}\,{{\theta}_{3}}\,{{\theta}_{4}}+{{b}_{13}}\,{{\theta}_{2}}\,{{\theta}_{4}}+{{b}_{14}}\,{{\theta}_{2}}\,{{\theta}_{3}}+{{b}_{23}}\,{{\theta}_{1}}\,{{\theta}_{4}}+{{b}_{24}}\,{{\theta}_{1}}\,{{\theta}_{3}}+{{b}_{34}}\,{{\theta}_{1}}\,{{\theta}_{2}} \\
&+{{b}_{12}}\,{{b}_{34}}+{{b}_{13}}\,{{b}_{24}}+{{b}_{14}}\,{{b}_{23}} ,\\ 
f_{1,1}&={{\theta}_{1}}\,{{\theta}_{2}}+{{b}_{12}}+{\rm e}^{{{\xi}_{3}}}\,{{\psi}_{13}}\,{{\psi}_{23}}+{\rm e}^{{{\xi}_{3}}}\,{{\psi}_{13}}\,{{\theta}_{2}}+{\rm e}^{{{\xi}_{3}}}\,{{\theta}_{1}}\,{{\psi}_{23}}+{\rm e}^{{{\xi}_{3}}}\,{{\theta}_{1}}\,{{\theta}_{2}}+{\rm e}^{{{\xi}_{3}}}\,{{b}_{12}} ,\\
f_{1,2}&={{\theta}_{1}}\,{{\theta}_{2}}+{{b}_{12}}+{\rm e}^{{{\xi}_{3}}}\,{{\psi}_{13}}\,{{\psi}_{23}}+{\rm e}^{{{\xi}_{3}}}\,{{\psi}_{13}}\,{{\theta}_{2}}+{\rm e}^{{{\xi}_{3}}}\,{{\theta}_{1}}\,{{\psi}_{23}}+{\rm e}^{{{\xi}_{3}}}\,{{\theta}_{1}}\,{{\theta}_{2}}+{\rm e}^{{{\xi}_{3}}}\,{{b}_{12}}\\ 
&+{\rm e}^{{{\xi}_{4}}}\,{{\psi}_{14}}\,{{\psi}_{24}}+{\rm e}^{{{\xi}_{4}}}\,{{\psi}_{14}}\,{{\theta}_{2}}+{\rm e}^{{{\xi}_{4}}}\,{{\theta}_{1}}\,{{\psi}_{24}}+{\rm e}^{{{\xi}_{4}}}\,{{\theta}_{1}}\,{{\theta}_{2}}+{\rm e}^{{{\xi}_{4}}}\,{{b}_{12}}\\
&+{{h}_{34}}\,{\rm e}^{{{\xi}_{3}}}\,{\rm e}^{{{\xi}_{4}}}\,{{\psi}_{13}}\,{{\psi}_{23}}+{{h}_{34}}\,{\rm e}^{{{\xi}_{3}}}\,{\rm e}^{{{\xi}_{4}}}\,{{\psi}_{13}}\,{{\psi}_{24}}+{{h}_{34}}\,{\rm e}^{{{\xi}_{3}}}\,{\rm e}^{{{\xi}_{4}}}\,{{\psi}_{13}}\,{{\theta}_{2}}+{{h}_{34}}\,{\rm e}^{{{\xi}_{3}}}\,{\rm e}^{{{\xi}_{4}}}\,{{\psi}_{14}}\,{{\psi}_{23}}\\
&+{{h}_{34}}\,{\rm e}^{{{\xi}_{3}}}\,{\rm e}^{{{\xi}_{4}}}\,{{\psi}_{14}}\,{{\psi}_{24}}+{{h}_{34}}\,{\rm e}^{{{\xi}_{3}}}\,{\rm e}^{{{\xi}_{4}}}\,{{\psi}_{14}}\,{{\theta}_{2}}+{{h}_{34}}\,{\rm e}^{{{\xi}_{3}}}\,{\rm e}^{{{\xi}_{4}}}\,{{\theta}_{1}}\,{{\psi}_{23}}+{{h}_{34}}\,{\rm e}^{{{\xi}_{3}}}\,{\rm e}^{{{\xi}_{4}}}\,{{\theta}_{1}}\,{{\psi}_{24}}\\
&+{{h}_{34}}\,{\rm e}^{{{\xi}_{3}}}\,{\rm e}^{{{\xi}_{4}}}\,{{\theta}_{1}}\,{{\theta}_{2}}+{{h}_{34}}\,{\rm e}^{{{\xi}_{3}}}\,{\rm e}^{{{\xi}_{4}}}\,{{b}_{12}} ,\\ 
f_{2,1}&={{\theta}_{1}}\,{{\theta}_{2}}\,{{\theta}_{3}}\,{{\theta}_{4}}+{{b}_{14}}\,{{\theta}_{2}}\,{{\theta}_{3}}+{{b}_{23}}\,{{\theta}_{1}}\,{{\theta}_{4}}+{{b}_{13}}\,{{\theta}_{2}}\,{{\theta}_{4}}+{{b}_{12}}\,{{\theta}_{3}}\,{{\theta}_{4}}+{{b}_{24}}\,{{\theta}_{1}}\,{{\theta}_{3}}+{{b}_{34}}\,{{\theta}_{1}}\,{{\theta}_{2}}\\
&+{{b}_{13}}\,{{b}_{24}}+{{b}_{14}}\,{{b}_{23}}+{{b}_{12}}\,{{b}_{34}}+{\rm e}^{{{\xi}_{5}}}\,{{b}_{12}}\,{{b}_{34}}+{\rm e}^{{{\xi}_{5}}}\,{{b}_{14}}\,{{b}_{23}}+{\rm e}^{{{\xi}_{5}}}\,{{b}_{13}}\,{{b}_{24}}+{\rm e}^{{{\xi}_{5}}}\,{{b}_{12}}\,{{\psi}_{45}}\,{{\theta}_{3}}\\
&+{\rm e}^{{{\xi}_{5}}}\,{{b}_{12}}\,{{\theta}_{3}}\,{{\theta}_{4}}+{\rm e}^{{{\xi}_{5}}}\,{{b}_{13}}\,{{\psi}_{25}}\,{{\psi}_{45}}+{\rm e}^{{{\xi}_{5}}}\,{{b}_{13}}\,{{\psi}_{25}}\,{{\theta}_{4}}+{\rm e}^{{{\xi}_{5}}}\,{{b}_{13}}\,{{\psi}_{45}}\,{{\theta}_{2}}+{\rm e}^{{{\xi}_{5}}}\,{{b}_{13}}\,{{\theta}_{2}}\,{{\theta}_{4}}\\
&+{\rm e}^{{{\xi}_{5}}}\,{{b}_{14}}\,{{\psi}_{25}}\,{{\psi}_{35}}+{\rm e}^{{{\xi}_{5}}}\,{{b}_{14}}\,{{\psi}_{25}}\,{{\theta}_{3}}+{\rm e}^{{{\xi}_{5}}}\,{{b}_{14}}\,{{\psi}_{35}}\,{{\theta}_{2}}+{\rm e}^{{{\xi}_{5}}}\,{{b}_{14}}\,{{\theta}_{2}}\,{{\theta}_{3}}+{\rm e}^{{{\xi}_{5}}}\,{{b}_{23}}\,{{\psi}_{15}}\,{{\psi}_{45}}\\
&+{\rm e}^{{{\xi}_{5}}}\,{{b}_{23}}\,{{\psi}_{15}}\,{{\theta}_{4}}+{\rm e}^{{{\xi}_{5}}}\,{{b}_{23}}\,{{\psi}_{45}}\,{{\theta}_{1}}+{\rm e}^{{{\xi}_{5}}}\,{{b}_{23}}\,{{\theta}_{1}}\,{{\theta}_{4}}+{\rm e}^{{{\xi}_{5}}}\,{{b}_{24}}\,{{\psi}_{15}}\,{{\psi}_{35}}+{\rm e}^{{{\xi}_{5}}}\,{{b}_{24}}\,{{\psi}_{15}}\,{{\theta}_{3}}\\
&+{\rm e}^{{{\xi}_{5}}}\,{{b}_{24}}\,{{\psi}_{35}}\,{{\theta}_{1}}+{\rm e}^{{{\xi}_{5}}}\,{{b}_{24}}\,{{\theta}_{1}}\,{{\theta}_{3}}+{\rm e}^{{{\xi}_{5}}}\,{{b}_{34}}\,{{\psi}_{15}}\,{{\psi}_{25}}+{\rm e}^{{{\xi}_{5}}}\,{{b}_{34}}\,{{\psi}_{15}}\,{{\theta}_{2}}+{\rm e}^{{{\xi}_{5}}}\,{{b}_{34}}\,{{\psi}_{25}}\,{{\theta}_{1}}\\
&+{\rm e}^{{{\xi}_{5}}}\,{{b}_{34}}\,{{\theta}_{1}}\,{{\theta}_{2}}+{\rm e}^{{{\xi}_{5}}}\,{{b}_{12}}\,{{\psi}_{35}}\,{{\psi}_{45}}+{\rm e}^{{{\xi}_{5}}}\,{{b}_{12}}\,{{\psi}_{35}}\,{{\theta}_{4}}+{\rm e}^{{{\xi}_{5}}}\,{{\theta}_{1}}\,{{\psi}_{25}}\,{{\theta}_{3}}\,{{\psi}_{45}}+{\rm e}^{{{\xi}_{5}}}\,{{\theta}_{1}}\,{{\psi}_{25}}\,{{\theta}_{3}}\,{{\theta}_{4}}\\
&+{\rm e}^{{{\xi}_{5}}}\,{{\theta}_{1}}\,{{\theta}_{2}}\,{{\psi}_{35}}\,{{\psi}_{45}}+{\rm e}^{{{\xi}_{5}}}\,{{\theta}_{1}}\,{{\theta}_{2}}\,{{\psi}_{35}}\,{{\theta}_{4}}+{\rm e}^{{{\xi}_{5}}}\,{{\theta}_{1}}\,{{\theta}_{2}}\,{{\theta}_{3}}\,{{\psi}_{45}}+{\rm e}^{{{\xi}_{5}}}\,{{\psi}_{15}}\,{{\psi}_{25}}\,{{\psi}_{35}}\,{{\psi}_{45}}\\
&+{\rm e}^{{{\xi}_{5}}}\,{{\psi}_{15}}\,{{\psi}_{25}}\,{{\psi}_{35}}\,{{\theta}_{4}}+{\rm e}^{{{\xi}_{5}}}\,{{\psi}_{15}}\,{{\psi}_{25}}\,{{\theta}_{3}}\,{{\psi}_{45}}+{\rm e}^{{{\xi}_{5}}}\,{{\psi}_{15}}\,{{\psi}_{25}}\,{{\theta}_{3}}\,{{\theta}_{4}}+{\rm e}^{{{\xi}_{5}}}\,{{\psi}_{15}}\,{{\theta}_{2}}\,{{\psi}_{35}}\,{{\psi}_{45}}\\
&+{\rm e}^{{{\xi}_{5}}}\,{{\psi}_{15}}\,{{\theta}_{2}}\,{{\psi}_{35}}\,{{\theta}_{4}}+{\rm e}^{{{\xi}_{5}}}\,{{\psi}_{15}}\,{{\theta}_{2}}\,{{\theta}_{3}}\,{{\psi}_{45}}+{\rm e}^{{{\xi}_{5}}}\,{{\psi}_{15}}\,{{\theta}_{2}}\,{{\theta}_{3}}\,{{\theta}_{4}}+{\rm e}^{{{\xi}_{5}}}\,{{\theta}_{1}}\,{{\psi}_{25}}\,{{\psi}_{35}}\,{{\psi}_{45}}\\
&+{\rm e}^{{{\xi}_{5}}}\,{{\theta}_{1}}\,{{\psi}_{25}}\,{{\psi}_{35}}\,{{\theta}_{4}}+{\rm e}^{{{\xi}_{5}}}\,{{\theta}_{1}}\,{{\theta}_{2}}\,{{\theta}_{3}}\,{{\theta}_{4}} .
\end{aligned}
\end{equation*}


\end{document}