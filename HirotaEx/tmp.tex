\documentclass{article}
\usepackage{ctex}
\usepackage{paralist}% 方便好用的压缩列表
\usepackage[a4paper,top=2cm,bottom=2cm,left=2cm,right=2cm,includehead,includefoot]{geometry}
\usepackage{amsmath,amssymb,amsfonts,bm}
\newcommand{\abs}[1]{\left\vert#1\right\vert}
\newcommand{\floor}[1]{\left\lfloor{#1}\right\rfloor}
\newcommand{\ceil}[1]{\left\lceil{#1}\right\rceil}
\newcommand{\sbrace}[1]{\left(#1\right)}
\newcommand{\mbrace}[1]{\left[#1\right]}
\newcommand{\bbrace}[1]{\left\{#1\right\}}
\newcommand{\dbrace}[1]{
  \Bigl\{
    #1
  \Bigr\} 
}
\newcommand{\eval}[2]{\left.{#1}\right|_{#2}}
\newcommand{\conj}[1]{{\rm conj}\sbrace{#1}}
\newcommand{\ALLP}{\mathcal{A}}
\newcommand{\PS}{\mathcal{P}}
\newcommand{\dd}[1]{\mathrm{d}#1}
\newcommand{\ii}[1]{\int\!{#1\dd x}}
\newcommand{\VecNorm}[1]{\left\Vert#1\right\Vert}% 向量模
\newcommand{\spell}[1]{#1}
\newcommand{\up}[1]{^{(#1)}}
\newcommand{\TT}{^\top}% 矩阵转置
\newcommand{\OO}{\ensuremath{\mathbb O}}% n 阶展开多项式余项
\newcommand{\OC}{\ensuremath{\mathcal O}}% 算法复杂度
\newcommand{\lfrac}[2]{#1/#2}
\newcommand{\DIF}[1]{\ensuremath{\frac{\partial}{\partial #1}}}
\newcommand{\DIFF}[2]{\ensuremath{\frac{\partial #1}{\partial #2}}}
\renewcommand{\mod}{{\rm ~mod~}}

\newcommand{\Painleve}{Painlev{\'e}}

\newcommand{\refeqn}[1]{Eq. (\ref{#1})}
\begin{document}
\title{Constructing soliton, breather, lump and their interaction solutions base on the extended Hirota method}
\author{余江涛}
\maketitle

\section{Extended Hirota Method}

对于一个$(d+1)$维的未知函数$u=u(x_1,\cdots,x_d,t)$, 关于它的 NLEE 是一个关于$u$及其导数的多项式方程, 即
\begin{equation}
    U(u,u\up{1},u\up{2},u\up{3}\cdots)=0, \label{oeq}
\end{equation}
其中$u\up{k}~(k=1,2,3,\cdots)$表示所有的$k$阶导数. 例如, $u\up{1}=\bbrace{u_t,u_{x_1},\cdots,u_{x_n}}$.

基于\Painleve{}展开法, 可以得到其 TPE
\begin{equation}
u=\sum_{k=1}^{\alpha}{\frac{u_k}{f^{\alpha-k+1}}}. 
\end{equation}
将其代入原方程可以得到一个关于$f$及其导数的多项式方程 
\begin{equation}
F\sbrace{f,f\up{1},f\up{2},\cdots}=0.
\end{equation}

对于$u=u\sbrace{x_1,\cdots,x_d,t}$, 其行波有多种假设形式. 例如, 
\begin{subequations}
\begin{align}
\xi&=p_1 x_1 + p_2 x_2 + \cdots + p_d x_d + \omega t+p_{d+1},  \label{xi-1}\\ 
\xi&=p_1 (x_1 + p_2 x_2 + \cdots + p_d x_d + \omega t)+p_{d+1}, \label{xi-2}\\ 
\xi&=\delta \sbrace{p_1 x_1 + p_2 x_2 + \cdots + p_d x_d + \omega t}+p_{d+1} \label{xi-3}. 
\end{align}
\end{subequations}
其中
\begin{compactitem}[\textbullet]
\item \refeqn{xi-1}只能用于求孤子解, 不能用于求 lump 解.
\item \refeqn{xi-2}和\refeqn{xi-3}可以用于求孤子解和lump解. 
\item 用\refeqn{xi-3}求解, 其形式更加复杂. 对于不可积方程, 其解成立的条件也更复杂.
\item \refeqn{xi-3}能够涵盖\refeqn{xi-1}和\refeqn{xi-2}, 是最一般的假设形式.
\end{compactitem}
所以我们还是用\refeqn{xi-3}作为行波的假设形式好了.

令
\begin{equation}
\PS\subseteq \ALLP=\bbrace{1,2,\cdots,d,d+1} ,
\end{equation}
我们定义下标替换函数
\begin{equation}
\mathcal F\sbrace{e,i;\PS}=\left\{\begin{array}{ll}
  \delta \to \delta_i, &  \\ 
  p_k \to p_{k,i}, & k \in \PS ,\\ 
  p_k \to p_k , & k \not\in\PS .
\end{array}\right.
\end{equation}

从$1+\exp(\xi)=0$中解得$\omega=\omega^*$, 可得 
\begin{equation}
\begin{aligned}
  \xi_i&=\mathcal{F}\sbrace{\eval{\xi}{\omega=\omega*},i;\PS} ,\\
  \xi_j&=\mathcal{F}\sbrace{\eval{\xi}{\omega=\omega*},j;\PS} .
\end{aligned}
\end{equation}
然后从$1+\exp(\xi_i)+\exp(\xi_j)+h_{i,j}\exp(\xi_i+\xi_j)=0$解得$h_{i,j}$. 

令
\begin{equation}
\begin{aligned}
M&=\bbrace{1,\cdots,2m}, \\ 
N&=\bbrace{2m+1,\cdots,2m+n}, \\ 
\end{aligned}
\end{equation}
设
\begin{equation}
\mathcal{L}(M)=\bbrace{\bbrace{s_1,\cdots,s_{2l}}|s_{2k}>s_{2k-1},s_{2k+1}>s_{2k-1},s_k\in M,l=0,\cdots,m} , 
\end{equation}
则 $m$-lump 和 $n$-孤子的相互作用解为
\begin{equation}
f_{m,n}=\sum_{T\subseteq N}\bbrace{
  \mbrace{
    \sum_{S\in \mathcal{L}(M)}{
      B(S)
      \Psi\sbrace{M-S,\bbrace{0}\cup T}
    }
  }
  \mbrace{
    \sbrace{\prod_{\bbrace{a,b}\subseteq T}{h_{a,b}}}
    \exp\sbrace{\sum_{k\in T}{\xi_k}}
  }
}. 
\end{equation}
其中 
\begin{equation}
\begin{aligned}
S&=\bbrace{s_1,\cdots,s_{|S|}}, \\ 
B(S)&=\prod_{k=1}^{|S|/2}{b_{s_{2k-1},s_{2k}}},\\ 
\Psi(S,T)&=\sum_{j_k\in T}{\prod_{k=1}^{|S|}{\psi_{s_k,j_k}}}, \\ 
\psi_{i,0}&=\theta_i . 
\end{aligned}
\end{equation}
需要说明的是, 当求和下标为空时, 结果为0; 当连乘下标为空时, 结果为1. 

还需满足 
\begin{equation}
\begin{aligned} 
\theta_i &= \eval{\DIFF{\xi_i}{\delta_i}}{\delta_i=0} ,\\ 
\psi_{i,j} &= \eval{\DIFF{h_{i,j}}{\delta_i}}{\delta_i=0} ,\\ 
b_{i,j} &= \eval{\frac{\partial^2 h_{i,j}}{\partial \delta_i \partial \delta_j}}{\delta_i=0,\delta_j=0} .
\end{aligned}
\end{equation}

令$n=2\kappa+\tau$, 取
\begin{equation}
p_{k,i}=\left\{
\begin{array}{ll}
  p_{k,i,RE}+I\cdot p_{k,i,IM}, & i\le 2(m+\kappa), i\equiv(1 \mod 2),\\
  p_{k,i-1,RE}-I\cdot p_{k,i-1,IM}, & i\le 2(m+\kappa), i\equiv(0 \mod 2),\\
  p_{k,i}, & 2(m+\kappa)+1 \le i \le 2(m+\kappa)+\tau ,
\end{array}
\right.
\end{equation}
\begin{equation}
\delta_i=\left\{\begin{array}{ll}
  \delta_{i,RE}+I\cdot\delta_{i,IM}, & 2m+1\le i \le 2m+2\kappa, i\equiv(1 \mod 2), \\ 
  \delta_{i-1,RE}-I\cdot\delta_{i-1,IM}, & 2m+1\le i \le 2m+2\kappa, i\equiv(0 \mod 2) ,\\
  \delta_i, & 2(m+\kappa)+1 \le i \le 2(m+\kappa)+\tau ,
\end{array}\right.
\end{equation}
可以得到$m$-lump, $\kappa$-呼吸子 和 $\tau$-孤子的相互作用解. 这里$m,\kappa,\tau\ge 0$, 所以取其中两个为零, 另外一个不为零也能得到单独的解. 最终, 解中所有的参数构成集合
\begin{equation}
\begin{aligned}
\mathbb{P}
&=\bbrace{\delta_i|2(m+\kappa)+1 \le i \le 2(m+\kappa)+\tau}  \\
&\cup\bbrace{\delta_{2i-1,RE},\delta_{2i-1,IM}|m+1\le i \le m+\kappa} \\ 
&\cup\bbrace{p_k|k\not\in \PS} \\
&\cup\bbrace{p_{k,2i-1,RE},p_{k,2i-1,IM}|k\in\PS,1\le i \le m+\kappa} \\
&\cup\bbrace{p_{k,i}|k\in\PS,2(m+\kappa)+1 \le i \le 2(m+\kappa)+\tau},
\end{aligned}
\end{equation}
且$\mathbb P$中所有的参数取值均为实数. 

\section{Examples}

\section{Proof}
需要证明, 当
\begin{equation}
\xi_k=\left\{\begin{array}{ll}
-\sbrace{1+\delta_k \theta_k}+o\sbrace{\delta_k}, & k\le 2m, \\ 
\xi_k, & k>2m,   
\end{array}\right.
\end{equation}
\begin{equation}
h_{i,j}=\left\{\begin{array}{ll}
1+\delta_i \delta_j b_{i,j}+o\sbrace{\delta_i \delta_j}, & i<j\le 2m, \\
1+\delta_i \psi_{i,j}+o\sbrace{\delta_i}, & i\le 2m < j , \\
h_{i,j}, & 2m<i<j,
\end{array}\right.
\end{equation}
时, 有
\begin{equation}
\begin{aligned}
f&=\sum_{T\subseteq \bbrace{1,\cdots,2m+n}}{\mbrace{
  \sbrace{\prod_{\bbrace{a,b}\subseteq T}{h_{a,b}}}
  \exp\sbrace{\sum_{k\in T}{\xi_k}} 
}} \\ 
&=\sbrace{\prod_{l=1}^m{\delta_k}}f_{m,n}+o\sbrace{\prod_{l=1}^m{\delta_k}}.
\end{aligned}
\end{equation}
其中$o(f)$是Peano余项, 满足$\lim_{f\rightarrow 0}[o(f)/f]=0$.

\end{document}