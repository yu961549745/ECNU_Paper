\documentclass{beamer}
\usetheme{CambridgeUS}
\usefonttheme{serif}
\setbeamertemplate{navigation symbols}{}

\usepackage{ctex}
\usepackage{amsmath,amssymb,amsfonts,bm}
\usepackage{graphicx,subfigure}
\usepackage{adjustbox}
\usepackage{color,xcolor}
\usepackage{tikz}
\usepackage{animate}

% 自定义数学公式
\newcommand{\abs}[1]{\left\vert#1\right\vert}
\newcommand{\floor}[1]{\left\lfloor{#1}\right\rfloor}
\newcommand{\ceil}[1]{\left\lceil{#1}\right\rceil}
\newcommand{\sbrace}[1]{\left(#1\right)}
\newcommand{\mbrace}[1]{\left[#1\right]}
\newcommand{\bbrace}[1]{\left\{#1\right\}}
\newcommand{\eval}[2]{\left.{#1}\right|_{#2}}
\newcommand{\conj}[1]{{\rm conj}\sbrace{#1}}
\newcommand{\ALLP}{\mathcal{A}}
\newcommand{\PS}{\mathcal{P}}
\newcommand{\dd}[1]{\mathrm{d}#1}
\newcommand{\ii}[1]{\int\!{#1\dd x}}
\newcommand{\VecNorm}[1]{\left\Vert#1\right\Vert}% 向量模
\newcommand{\spell}[1]{#1}
\newcommand{\up}[1]{^{(#1)}}
\newcommand{\TT}{^\top}% 矩阵转置
\newcommand{\OO}{\ensuremath{\mathbb O}}% n 阶展开多项式余项
\newcommand{\OC}{\ensuremath{\mathcal O}}% 算法复杂度
\newcommand{\lfrac}[2]{#1/#2}
\newcommand{\DIF}[1]{\ensuremath{\frac{\partial}{\partial #1}}}
\newcommand{\DIFF}[2]{\ensuremath{\frac{\partial #1}{\partial #2}}}
\newcommand{\cd}[1]{\,\texttt{#1}\,}
\newcommand{\dbrace}[1]{
  \Bigl\{
    #1
  \Bigr\} 
}

\title[华东师范大学硕士学位论文]{非线性系统的精确解及其\\ 基础算法的符号计算研究}
\author[余江涛]{余江涛 \and \\导师: 柳银萍~教授}
\date{\today}

\begin{document}
\frame{
    \tikz[overlay,remember picture]\node[opacity=0.1]at (current page.center){\includegraphics[width=0.7\paperheight]{../paper/sty/ecnu_logo.pdf}};
    \titlepage
}

\section{本文工作}
\begin{frame}
\frametitle{本文工作}
\begin{figure}
\centering
\includegraphics[height=0.8\textheight]{fig/ppt-outline.pdf} 
\end{figure}
\end{frame}

\section{三种波解}
\begin{frame}{孤子解}
\begin{figure}
\centering
\subfigure[扭状孤子]{
    \animategraphics[loop,autoplay,width=0.45\textwidth]{8}{fig/(2+1)BKP-T-001/}{001}{024}
}    
\subfigure[钟状孤子]{
    \animategraphics[loop,autoplay,width=0.45\textwidth]{8}{fig/(2+1)BKP-001/}{001}{024}
}    
\end{figure}
\end{frame}

\begin{frame}
\begin{figure}
\animategraphics[loop,autoplay,width=0.6\textwidth]{8}{fig/(2+1)BKP-003/}{001}{024}
\caption{3-孤子解}
\end{figure}
\end{frame}

\begin{frame}{呼吸子解}
\begin{figure}
\setcounter{subfigure}{0}
\subfigure[扭状孤子对应的呼吸子解]{
  \animategraphics[loop,autoplay,width=0.45\textwidth]{8}{fig/(2+1)BKP-T-010/}{001}{024}
}
\subfigure[钟状孤子对应的呼吸子解]{
  \animategraphics[loop,autoplay,width=0.45\textwidth]{8}{fig/(2+1)BKP-010/}{001}{024}
}
\end{figure}
\end{frame}

\begin{frame}
\begin{figure}
\animategraphics[loop,autoplay,width=0.6\textwidth]{8}{fig/(2+1)BKP-020/}{001}{024}
\caption{2-呼吸子解}
\end{figure}
\end{frame}

\begin{frame}{呼吸子解取极限}
\begin{figure}
\setcounter{subfigure}{0}
\subfigure[呼吸子取极限]{
  \animategraphics[loop,autoplay,width=0.45\textwidth]{8}{fig/(2+1)BKP-delta-010/}{001}{024}
}
\subfigure[lump 解]{
  \includegraphics[width=0.45\textwidth]{fig/(2+1)BKP-delta-010/lump.png}
}
\end{figure}
\end{frame}

\begin{frame}{lump解}
\begin{figure}
\setcounter{subfigure}{0}
\subfigure[扭状孤子对应的lump解]{
  \animategraphics[loop,autoplay,width=0.45\textwidth]{8}{fig/(2+1)BKP-T-100/}{001}{024}
}
\subfigure[钟状孤子对应的lump解]{
  \animategraphics[loop,autoplay,width=0.45\textwidth]{8}{fig/(2+1)BKP-100/}{001}{024}
}
\end{figure}
\end{frame}
\begin{frame}
\begin{figure}
\animategraphics[loop,autoplay,width=0.6\textwidth]{8}{fig/(4+1)Fokas-T-2-300/}{001}{024}
\caption{3-lump解}
\end{figure}
\end{frame}

\section{相互作用解}
\begin{frame}
\begin{figure}
\animategraphics[loop,autoplay,width=0.6\textwidth]{8}{fig/(3+1)YTSF-100/}{001}{024}
\caption{(3+1)YTSF 方程的 0S-1L 解}
\end{figure}
\end{frame}
\begin{frame}
\begin{figure}
\animategraphics[loop,autoplay,width=0.6\textwidth]{8}{fig/(3+1)YTSF-101/}{001}{024}
\caption{(3+1)YTSF 方程的 1S-1L 解}
\end{figure}
\end{frame}
\begin{frame}
\begin{figure}
\animategraphics[loop,autoplay,width=0.6\textwidth]{8}{fig/(3+1)YTSF-102/}{001}{024}
\caption{(3+1)YTSF 方程的 2S-1L 解}
\end{figure}
\end{frame}

\section{n阶展开方法}

\section{积分化简}

\section{致谢}
\begin{frame}
\tikz[overlay,remember picture]\node[opacity=0.2]at (current page.center){\includegraphics[width=0.7\paperheight]{../paper/sty/ecnu_logo.pdf}};
\centerline{\Huge 谢谢}
\end{frame}
\end{document}